\newcommand{\subtitulo}[1]{\NoCaseChange{\textnormal{\break\Large\itshape#1}}}
\chapter*{Apresentação\smallskip\subtitulo{Um \emph{road movie} em forma\\de narrativa}}
\markboth{Apresentação}{Lilian Starobinas}
\addcontentsline{toc}{chapter}{Apresentação, \textit{por Lilian Starobinas}}


\begin{flushright}
\textsc{lilian starobinas}
\end{flushright}\medskip

\noindent{}Das páginas manuscritas em iídiche, agora amareladas pelo tempo, numa
caligrafia miúda e regular, provém essas histórias que compõem a
autobiografia de meu avô. Foram registradas por ele no ano de 1964, num
caderno que se tomou um tesouro familiar. Seu Jaime, como o chamávamos,
viveu ainda mais duas décadas, mas nunca se preocupou em escrever suas
memórias em português. Contá-las oralmente sim, ele que era um grande
contador de histórias, bom de papo, comunicativo. Assim, muitas das
passagens que publicamos agora, traduzidas, fazem parte do repertório que
escutávamos o vovô contar, e que foram construindo nossa memória sobre
as origens da família.

\section{sobre a tradução}

Foi a filha mais nova, Cecília, minha mãe, quem tomou para si o desafio
de publicar o livro, e assim trazer a público as narrativas sobre a
trajetória de imigração e estabelecimento no Brasil de seu pai. Mais que
uma homenagem, ela teve consciência da singularidade desse material e da
contribuição que ele traz para a história dos processos de imigração.

Ao longo dos anos, ela empenhou-se para conseguir quem traduzisse.
Inicialmente, investiu num processo artesanal, com a ajuda da sogra,
minha avó Rosa Starobinas, que lia em voz alta o texto e traduzia os
trechos na sequência, enquanto minha mãe registrava sua versão no
gravador com fita cassete. Sem sucesso em fazer a tradução por completo,
decidiu procurar quem fizesse profissionalmente, e assim chegou em Léa
Baran. Entre a tradução e a publicação passaram-se ainda alguns anos.
Finalmente conseguimos compartilhar com mais leitores essas saborosas
histórias de Chaim Novodvorsky.

\section{o contexto histórico}

A Polônia é seu lugar de origem. Viviam ali, estima-se, mais de
3 milhões de judeus antes da Segunda Guerra Mundial. A comunidade
judaica de Knishin, o \textit{shtetl}\footnote{Do iídiche, ``cidadezinha''. O termo se aplica a povoações ou bairros de cidades com população predominantemente judaica.} onde nasceu o autor deste livro, contava com cerca de 3.500 membros em 1900, e 1.235 em 1921. Perto dali ficava Bialistok, uma comunidade muito maior, que passava dos 40 mil habitantes judeus na
década de 1920. 

A intensificação do processo de emigração se deu a partir das primeiras décadas do século \textsc{xx}, diante das duras condições econômicas, dos riscos físicos representados pelos
\textit{pogroms},\footnote{Termo usado para descrever um ataque violento massivo, com destruição de casas, negócios, centros religiosos. Atribuído à perseguição deliberada de um grupo étnico ou religioso, aprovado ou tolerado pelas autoridades locais.} das destruições e horrores frutos da Primeira Guerra Mundial,
das incertezas provocadas pela eclosão da Revolução Russa e das
instabilidades políticas por toda a Europa. Tornou-se frequente, dentre os
judeus da Europa Oriental, o desejo de estabelecerem-se nos Estados
Unidos, país visto como terra das oportunidades e da liberdade de
religião, dando origem à expressão \textit{fazer a América}. Paralelamente estruturava-se o movimento sionista, que estimulava a imigração para a Palestina. O desejo de construir um lar nacional judaico passou a ser defendido como necessário para a uma existência autônoma e livre das perseguições vivenciadas por judeus em diferentes países da Europa ao longo da história. 

Entre 1880 e 1914, partiram cerca de dois milhões de judeus
asquenazitas, provenientes da Europa Oriental, de regiões que formavam o
Pale (zonas de residências permitidas ao judeus). Trata-se de uma área
situada em terras pertencentes ao Império Russo, ao Império
Austro-Húngaro, à Prússia e à Romênia. Chegaram aos Estados Unidos.
Falantes do iídiche, representaram notória influência na cultura local,
na produção literária e nos campos da música, do teatro, do cinema, das
artes em geral. A partir de 1921, os Estados Unidos estabeleceram uma
lei de emergência limitando a imigração, provocada pelo expressivo
aumento de imigrantes após o final da Primeira Guerra Mundial, momento
em que se instaurou uma profunda recessão econômica na Europa.\looseness=-1

A imigração para a América do Sul colocou-se como uma alternativa às
restrições norte-americanas. Desde o final do século \textsc{xix}, ampliou-se a
adoção da Argentina como porto de destino, e estima-se que, até 1920, o
número de imigrantes judeus tenha chegado a cerca de 150 mil. Parte
desse contingente estabeleceu-se em colônias agrícolas, criadas pela
Jewish Colonization Association, uma iniciativa filantrópica voltadas à
absorção de imigrantes e sua inserção econômica nos países de destino.
Uruguai, Brasil, Paraguai e Chile, entre outros países da região,
receberam imigrantes judeus, e viveram o processo de formação de
comunidades locais, com instituições voltadas ao atendimento das
necessidades desses imigrantes e espaços para formação educacional e
religiosa, produção cultural e sociabilidade, sociedades voltadas aos
cuidados de saúde e de sepultamento.

A narrativa de \textit{Em busca de meus irmãos na América} nos permite 
saber mais sobre aspectos dessas redes
de solidariedade formada por imigrantes com referenciais culturais
similares, dando a conhecer não só os sucessos dessas relações, mas
também suas tensões. Encontramos nela as disputas por trabalho, os
estelionatários vendendo ilusões, episódio de discriminação entre os
próprios imigrantes, compondo uma história pouco preocupada com
romantizar a trajetória pessoal. Também se fazem presentes 
os acolhimentos e a ajuda mútua, a valorização dos esforços e do
caráter, as oportunidades de integração e de melhoria das condições de
vida. Chegando na Argentina, no começo dos anos 1920, Chaim deposita
seus esforços no projeto de ir ao encontro dos irmãos, que imigraram
para os Estados Unidos. E assim vai construindo seu percurso em direção
ao norte, escolhendo pontos intermediários nessa rota até seu destino
ideal. Esse traçado dá à narrativa um clima de \textit{road-movie}, pelos
encontros inusitados, a precariedade das situações, a diversidade de
cultura e costumes com as quais ele se depara, a insistência em cumprir
seu destino autoproclamado.\looseness=-1

No coração dessa jornada se encontra o Brasil, e parte importante dessas
memórias está ligada a este país. Mais de quatro milhões de imigrantes
somaram-se à população brasileira, de 1872 até 1940, e entre eles
estima-se a chegada de 60 mil judeus. A imigração judaica deu origem a
comunidades no Rio de Janeiro e em São Paulo, mas também em Porto Alegre,
Curitiba, Belo Horizonte, Salvador, Recife, Belém, Manaus, além de núcleos
menores fora das capitais. Intensificou-se nos anos 1920, período em que a
politica imigratória dos Estados Unidos sofreu um revés.\looseness=-1

Após cem anos de sua chegada, estima-se que há 120 mil judeus no país. Instituições de apoio,
escolas, clubes, sinagogas, escritores e círculos de leitura, teatros,
corais, conjuntos musicais, restaurantes, uma ampla e rica diversidade
de produções e convivências permeiam a vida judaica hoje no Brasil.
Entre associações e disputas, em meio a tempos mais brandos ou mais
difíceis no que diz respeito à vida política e econômica, seguem
compondo comunidades que exercitam o acolhimento ao estrangeiro, valor
de enorme importância na tradição judaica.

