%\newcommand{\subtitulo}[1]{\NoCaseChange{\textnormal{\break\Large\itshape#1}}}
\chapter*{Posfácio\smallskip\subtitulo{Os percursos da imigração judaica}}
\markboth{Posfácio}{Roney Cytrynowicz}
\addcontentsline{toc}{chapter}{Posfácio, \textit{por Roney Cytrynowicz}}

\begin{flushright}
\textsc{roney cytrynowicz}
\end{flushright}\medskip

\noindent{}\textit{Em busca de meus irmãos na América}, de Chaim
Novodvorsky, é um texto de memória que prende o leitor do começo ao fim.
É um relato pessoal, singular, e, ao mesmo tempo, emblemático dos
percursos da imigração judaica. Mescla de forma saborosa os
acontecimentos e as aventuras pessoais de um imigrante, que foi primeiro
à Argentina, depois ao Uruguai e, finalmente, ao Brasil, com um preciso
e vívido retrato dos caminhos pelos quais se dava a inserção dos
imigrantes na vida do país entre os anos 1920 e 1960.

``Fiquei triste e solitário. Percebi que daí para frente estaria
só no mundo, muito só, e não teria mais ninguém da minha família por perto'', escreveu Chaim sobre a decisão de deixar a
Polônia em 1922 e rumar para Buenos Aires, de onde, após uma passagem
por Montevidéu, chegaria ao Brasil em 1928. 

\section{a decisão de emigrar}

Processo socialmente complexo a decisão de emigrar, resultado de muitos fatores, mas tendo
como motriz a expectativa de um futuro promissor e de um horizonte de
esperança. No caso de Chaim, parte dos irmãos já tinha imigrado para a
América e ele também fugia do serviço militar, um dos mais frequentes
motivos de emigração da Europa Oriental e do Império Russo desde o
século \textsc{xix}. Um relato como o de Chaim repõe a dimensão pessoal, afetiva, do
significado de um jovem, aos 19 anos, sem profissão, deixar a família em
um pequeno vilarejo na Polônia com o objetivo de imigrar aos Estados
Unidos e, diante da recusa de visto àquele país, lançar-se sozinho rumo
ao outro lado do mundo. ``Desembarquei em Buenos Aires, quase não tinha
dinheiro, não sabia falar o idioma. Soube que existia a Casa dos
Imigrantes, pertencente ao governo'', escreveu ele.

As memórias de Chaim nos revelam alguns dos fatores estruturais que
caracterizaram a imigração judaica ao Brasil. Foi nos anos 1920, com as
barreiras impostas à imigração em países como Estados Unidos e Canadá,
que a imigração ao Brasil se adensou, o país se tornou um horizonte
possível e desejável para os emigrantes, como mostrou o historiador
Jeffrey Lesser. Entre 1921 e 1925, segundo o demógrafo Jacob
Letschinsky, a Argentina foi o terceiro país a receber imigrantes
judeus, quase 40 mil pessoas ante 280 mil para os Estados Unidos, 61 mil
para a Palestina e 7.100 para o Brasil. No período 1926--1930, o Brasil
receberia 22.200 mil judeus, os \textsc{eua}, 55 mil, e a Argentina, 33.700.

No Brasil, ocorreu a formação mais ou menos simultânea de comunidades
judaicas em nada menos do que dez capitais do país a partir dos anos
1910 e consolidadas nos anos 1920: Porto Alegre e Curitiba na região
Sul, Rio de Janeiro, São Paulo e Belo Horizonte no Sudeste, Manaus e
Belém no Norte e Recife, Natal e Salvador no Nordeste, sendo que as
comunidades de Belém e do Rio de Janeiro foram formadas no século \textsc{xix}.

Imigração, urbanização e comunidade estavam diretamente associados na 
imigração judaica ao país. Em sua busca pelo encontro dos
irmãos na América, Chaim percorre o Brasil rumo ao Norte e Nordeste, e passa por pelo menos cinco das capitais nas quais existiam 
comunidades: Rio de Janeiro, onde chegou, depois Recife, Belém, Manaus e
Natal ---, além de Campina Grande, João Pessoa e Olinda.

A acelerada urbanização, em especial nas capitais e nas grandes cidades,
propiciou oportunidades de trabalho e de negócios, de inserção e de
ascensão social. Esse processo foi acentuado para um grupo de imigrantes
urbanos, o que vale para os judeus da Europa Oriental, da Europa Central
e do Oriente Médio, com experiências em trabalhos urbanos, funções
ligadas ao comércio e ofícios diversos. Ao chegar às grandes cidades do
país, mesmo os que não tinham uma profissão, um ofício definido, como
Chaim, encontraram sustento, trabalho e oportunidades de pequenos
negócios, ainda que, como mostra Chaim, muitos trabalhos remuneravam com
\textit{cama e comida} apenas.

\section{a inserção social e econômica}

A inserção social e econômica não foi fácil, como estas memórias
mostram extensamente e de forma muito autêntica. Nem foi feita a partir de uma linha
reta, rumo ao sucesso. Foi um caminho cheio de percalços, situações de
indevida exploração, pequenos sucessos, pequenos fracassos, sucessos
maiores, fracassos maiores, mas ao mesmo tempo cheio de possibilidades
de sustento e de oportunidades e brechas de inserção econômica. 
% O relato
% direto e franco de Chaim sobre seus vários trabalhos, situações difíceis
% e suas várias iniciativas e empreendimentos, torna estas memórias
% singulares.

A chegada dos imigrantes a grandes cidades, onde formaram comunidades,
também se beneficiou de um ambiente mais aberto e tolerante, que
facilitou a inserção e a aceitação dos recém-chegados --- mesmo
convivendo com uma multiplicidade de estereótipos positivos e negativos,
mas não de um antissemitismo que significasse segregação e racismo
aberto com barreiras à inserção e ascensão (como havia, de vários tipos,
na vida judaica na Europa Oriental). Por isso, nas memórias
autobiográficas de Chaim este tema é praticamente ausente, não mereceu
registro nenhuma forma de discriminação, nem mesmo o de algum eventual
estereótipo ou preconceito verbalizado.

A formação de comunidades judaicas com sua sólida vida institucional deu
o entorno e a segurança basilar, em uma sociedade inteiramente carente
da presença do Estado em educação, saúde e assistência social. As
entidades das comunidades judaicas em cada capital proviam recepção aos
imigrantes, atendimento em saúde, escola, apoio assistencial para homens
e mulheres, lar de crianças, cooperativa de crédito, também organizações
e atividades culturais, políticas, sociais, esportivas e recreativas,
além de grupos de jovens e de mulheres e também cemitério e
sepultamento. Assim, além de contorno para preservar a identidade
grupal, as instituições foram uma poderosa alavanca social que permitiu
a inserção e ascensão de parcelas expressivas do grupo.

No texto de Chaim encontramos diversas menções às instituições de apoio
aos recém-chegados, na Argentina e no Brasil, como a casa dos
imigrantes, a cozinha e cantina comunitária, a imprensa ídiche com seus
anúncios de emprego, a cooperativa, a escola, mas também as pensões e
uma rede informal de recepção, amparo e depois mecanismos de inserção
econômica, ainda que muitas vezes com propostas de trabalho indignas,
como conta o autor.

\section{a vida comunitária}

Em cada cidade na qual chegava, Chaim procurava seus conterrâneos, ouvia
conselhos, se associava e empreendia seus negócios. Ao chegar à capital
uruguaia: ``Desembarquei em Montevidéu pela manhã e no porto comecei a
perguntar onde moravam os judeus da cidade. Me disseram que não muito
longe do cais havia um salão de barbearia, cujo dono era judeu''. No dia
seguinte após chegar ao chegar ao Rio de Janeiro, escreve: ``De manhã
fui procurar onde ficavam os judeus, andei e de repente estava em plena
Praça Onze, ali encontrei muitos judeus, comecei a perguntar `Onde posso
achar um quarto para morar', me mostraram uma pensão onde uma senhora
viúva alugava quartos e fornecia refeições''. 

O mesmo se deu em Recife,
onde Chaim comparecia diariamente à Praça Maciel Pinheiro, centro da
vida comunitária, para encontrar conterrâneos. Chama a atenção a
mobilidade geográfica, para além da iniciativa pessoal de Chaim, que
rumava para o norte, os Estados Unidos. A estrutura comunitária
propiciava e acolhia estes deslocamentos, que fortaleciam a formação de
redes de sociabilidade, economia, relacionamentos e casamentos, como de
fato ocorreu com Chaim ao ser apresentado a uma moça e casar-se.

Tanto os processos de urbanização que proporcionaram oportunidades de
trabalho e de inserção, em ambientes mais abertos, como o da estrutura
comunitária, cujas instituições promoveram o patamar básico das
necessidades cotidianas, nada têm de naturais, são mecanismos complexos
que o texto de Chaim permite entender com precisão.

\section{um anti-herói}

A narrativa de Chaim é o relato da vida real dos encontros e
desencontros, sucessos e fracassos, ganhos e perdas, solidariedades,
generosidades, explorações indevidas, trapaças de sócios e assim por
diante. Um dos traços que torna essa narrativa tão singular é a
espontaneidade de um imigrante que não se envergonha de contar seus
fracassos. O que acaba contrastando com um gênero de memórias,
depoimentos e autobiografias que transforma a vida dos imigrantes em uma
saga individual e liberal de vencedores, o que às vezes até projetos de
história oral acabam acentuando, assumindo uma narrativa que não
enxergava com clareza a rede informal e de instituições de acolhimento e
de oportunidades e o contexto maior que propiciou a inserção e eventual
ascensão social.

Como as memórias de Chaim mostram em diversas passagens, que ocupam o
centro de suas memórias, o trabalho de mascate, e depois no comércio e
como lojista e importador, exigia sutis percepções sobre as tramas
urbanas e complexos aprendizados específicos, e também uma rede de apoio
para o crédito e mercadorias iniciais e a passagem dos conhecimentos
sobre onde mascatear, o que vender, como vender, como cobrar e assim por
diante. O mascate, emblema do trabalho dos imigrantes, não era uma
figura pitoresca e transitória no processo imigratório, era o meio
efetivo de inserção econômica e social dos recém-chegados e muitos
ficaram anos e décadas no ofício, experiência comum a imigrantes
portugueses, sírios, libaneses, judeus e outros, transmitindo seus
saberes e práticas aos recém-chegados.

Depois de diversas iniciativas, Chaim empreende diversos negócios, entre
eles um autêntico e pioneiro negócio de \textit{delivery}, por meio do qual
vai à casa das pessoas e anota seus pedidos para entregar de carroça no
dia seguinte. Perceber, conceber e implementar negócios assim não é
trivial, é uma soma de aprendizados e percepções do emaranhado da vida
urbana. Outros trabalhos que ele exerceu eram sucedâneos deste, como o
de ambulante com realejo, ofícios que propiciavam transitar na
informalidade e com a mobilidade que a situação requeria. E depois, já
estabelecido, no comércio e da importação de mercadorias, como
automóveis nos anos 1940.

No Recife, já casado, ao estabilizar-se economicamente, ``tornei-me uma
pessoa importante, proprietário, e podia mandar os meus filhos para uma
escola judaica para estudarem'', escreveu ele. Por entender
perfeitamente a importância dessas instituições e rede de apoio em sua
própria trajetória é que podemos entender o orgulho com o qual Chaim
conta da sua participação, como tesoureiro, e na tomada de uma decisão
importante sobre o imóvel, em uma escola judaica em Recife durante a
Segunda Guerra Mundial. Nesse período, pôde contribuir, orgulhoso, com a
comunidade. Aliás, Chaim cita nominalmente os professores de língua
ídiche na escola de Recife, sinal da importância que ele dava à passagem
geracional da língua e da cultura dos judeus da Europa Oriental.

As memórias de Chaim perpassam ainda temas importantes da história da
imigração judaica e da história do Brasil, como a passagem por
Mosesville, colônia agrícola fundada pela Jewish Colonization
Association na Argentina; a atuação do teatro ídiche, quando vivia na
Argentina, incluindo a participação na encenação da peça \textit{Iberguss},
baseada na história das polacas no Brasil, e a Segunda Guerra Mundial,
quando estava em Recife e conviveu com os soldados das Forças Armadas
dos \textsc{eua} e participou da campanha de doação de aviões à recém-criada
Força Aérea Brasileira.

Os encontros com os irmãos nos \textsc{eua}, após tantos anos de
separação, são comoventes, bem como a posterior mudança para São Paulo,
os empreendimentos na capital paulista, inclusive um cinema no Bom
Retiro, negócios no ramo têxtil, e os acontecimentos que encerram o
período deste relato e fecham as memórias.

Este texto se insere em uma linha de valorização da
memória, que permite uma diversa e produtiva riqueza de
leituras. No caso da imigração judaica, alguns dos memorialistas se tornaram clássicos, como
Samuel Malamud e seu \textit{Recordando a Praça Onze} --- a praça onde Chaim esteve, ao chegar
ao Rio de Janeiro. Mas tem também uma linda história familiar, de passar a história 
às próximas gerações. E a decisão por sua publicação, praticamente cem anos após a emigração de Chaim, gera ainda novos sentidos familiares e afetivos a partir do estatuto público que ele ganha.
 %e privilegia o relato subjetivo

% Ao decidir registrar suas memórias, Chaim legou à sua família um
% presente precioso, relatando com autenticidade e espontaneidade sua
% trajetória e imprimindo muitos sentidos à sua vida a partir da imigração
% e do processo de busca de reunir-se aos irmãos. Ao decidir tornar
% pública estas memórias, a sua família deu um presente aos leitores,
% ampliando a circulação e as possibilidades de leitura e compreensão
% deste texto singular.