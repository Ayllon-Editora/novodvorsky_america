\chapter*{}
\thispagestyle{empty}

\mbox{}
\vfill
\begin{flushright}
\textit{Meu nome é Chaim Novodvorsky.\\ 
Nasci em Knishin, uma pequena cidade\\ 
da Polônia, perto de Bialistok, em 1903.\\ 
Após viver 42 anos viajando pela Argentina,\\ 
Uruguai e Brasil, onde me estabeleci,\\ 
resolvi escrever a minha autobiografia.}
\end{flushright}

\part[Em busca de meus irmãos na América]{Em busca de meus\\irmãos na América}

\chapter{A partida da Polônia}

\noindent{}Em 1922, com apenas 19 anos, recebi o meu passaporte e resolvi emigrar
para os Estados Unidos da América, onde meus irmãos já moravam
há algum tempo. Eu queria me reunir a eles. Viajei para Varsóvia, com
a certeza de obter um visto do consulado americano. Mas meu pedido 
foi recusado: me explicaram 
que a quota de imigração permitida pelos Estados Unidos para 
aquele ano já estava preenchida.

\textls[-20]{Fui então até uma companhia marítima, onde me informaram que seria possível
viajar para a Argentina. De lá poderia tentar um visto, dizendo que já residia ali há cinco anos. Desse modo, conseguiria ir para a América.}\looseness=-1

Voltei para casa e contei tudo ao meu pai. Resolvemos que eu
iria então para a Argentina, e novamente fui a Varsóvia para 
alterar o destino de meu passaporte.

\textls[10]{Despedi-me de meu pai, irmãos, irmãs, cunhadas, tios e tias em
segredo. Ninguém no vilarejo podia saber que eu estava de partida, já tinha 
sido convocado para servir no exército.}%, então ocultamos a despedida.

\textls[10]{No dia seguinte, fui até o cemitério para me despedir da minha querida
mãe, já falecida. À noite, discretamente, fui à estação para
tomar o trem para Bialistok, onde embarcaria para Varsóvia. Encontrei-me lá 
com minha cunhada Bella, de quem me despedi. E entrei no trem.}

%Preencher nota. \footnote{Segundo a tradição (...)}
\textls[10]{Recebi meu visto no consulado argentino e preparei tudo para
viajar. Aguardava apenas meu pai, que viria até Varsóvia se despedir de mim.
Ele era muito religioso e só comia \textit{kasher}, então, após sua chegada, nos
alimentamos apenas de comida láctea, pão,
manteiga e chá. Entendi que ele não iria comigo até a estação porque
eu viajaria no sábado, e o convenci a voltar para casa na
noite da quinta-feira. }

Quando tomei o trem para a Alemanha me senti muito só. Via que as pessoas embarcando no mesmo vagão se despediam de seus parentes. Fiquei triste e solitário. Percebi que daí para frente estaria
só no mundo, muito só, e não teria mais ninguém da minha família por perto.

Chegando à estação alemã, embarquei direto para a França, e de lá peguei o navio que me
levou até Buenos Aires, na Argentina. Cheguei em 22 de junho de 1922. A viagem durou 21 dias.

\chapter{Chegada à Argentina}
%\addcontentsline{toc}{chapter}{Os primeiros anos na Argentina}
%\markboth{Os primeiros anos\ldots}{}

\textls[10]{Desembarquei em Buenos Aires. Quase não tinha dinheiro e não sabia falar
o idioma. Soube da existência da Casa dos Imigrantes, uma instituição governamental.}

\textls[-15]{Tinha trazido comigo o endereço de uma senhora cujos pais eram nossos
vizinhos em Knishin. Não a conhecia, ela já morava há muitos anos
na Argentina. Tomei o bonde e fui procurá-la, a partir do endereço que seus pais me deram. E encontrei! ``Sou filho de Elie Novodvorsky, e neto de Chaim Shneguevitch'', me
apresentei. Logo ela entendeu quem eu era.}\looseness=-1

\textls[-20]{O casal tinha sete filhos, três meninas e quatro meninos. O filho mais
velho era casado. Todos eles moravam numa sala: quer dizer, num quarto
grande, dividido com um biombo. Na frente havia uma loja, onde vendiam cigarros, e ao lado um pequeno espaço, alugado por um
barbeiro. À noite, a família colocava camas para o casal e também para as
crianças, separadas pelo biombo. Logo vi que não teria lugar para
mim ali.}\looseness=-1

\textls[-20]{Fui então para a Casa dos Imigrantes, já que tinha permissão de dormir
lá por 30 dias. Embora não tivesse profissão, comecei a procurar qualquer
coisa, para sobreviver. Comprei um jornal iídiche e vi um anúncio:
``Precisa-se um padeiro''. Pensei: ``deve ser um trabalho bom'', achei
que não seria pesado. Imagine, só misturar a farinha com água. Lembrei que sempre observava nosso vizinho fazendo o pão, e ele era um
bom padeiro.}\looseness=-1

Dirigi-me ao endereço indicado no anúncio e me apresentei como padeiro.
 O dono da padaria chamava-se Chaikel e o estabelecimento ficava na rua
 Junin, entre as avenidas Corrientes e Lavalle. Ele pediu que voltasse ao
 anoitecer para fazer o teste. Era sexta-feira.

\textls[15]{Quando cheguei lá, me perguntou se eu sabia sovar a massa. Respondi que na minha cidade não
tínhamos o hábito de sovar, e ele logo viu que eu não era padeiro.
Mesmo assim me deixou ficar, para que não precisasse andar de noite pelas ruas que
ainda não conhecia. Ajudei a lavar as assadeiras e formas usadas para pães salgados e doces. Ele me ofereceu chá-mate, tão doce e denso que até grudou nos
meus lábios. Eu nunca tinha provado. Aceitei e não disse nada, porque
que era um \textit{griner},}\footnote{Imigrante recém chegado.} e não queria
que rissem de mim.

\textls[15]{Trabalhei a noite inteira, dormi um pouco e de madrugada ele me acordou
 e disse: ``Meu jovem, você não é padeiro'', ao que eu retruquei: ``Preciso
 trabalhar, porque não tenho ninguém aqui. Estou na Casa do Imigrante e
 tenho que ganhar o meu sustento''. Ele disse então que eu poderia ficar
 e trabalhar como peão: eu nem sabia o que era isso, mas aceitei. Depois
 fiquei sabendo que \textit{peão} é uma pessoa que faz todo tipo de serviço
 que precisa ser feito na padaria.} %Na minha terra, isso se chamava \textit{carregador}.}

\textls[-7]{No sábado de manhã chegou um carregamento de farinha. Tive que descarregar
 rápido, só que em vez de colocar nas costas, como na minha terra,
 jogaram em cima do ombro, com tanta força, que quase caí. Nem podia
 reclamar, pois ainda não falava espanhol. Quando terminei com a farinha 
 chegou um vagão de sal, e comecei novamente a
 descarregar. O sal estava molhado e pesava muito. Foi colocado em
 grandes cestas lotadas, e eu de novo não podia falar nada, pois não
 sabia como dizer isso em espanhol. Terminei o trabalho ao meio dia. E das duas horas da tarde até as cinco, entreguei pão preto nas pequenas mercearias.}\looseness=-1

%para as pessoas que entregavam no veículo: retirado.
%além da entrega do açúcar para os padeiros: não entendi.
\textls[-20]{Continuei trabalhando na padaria, com direito a
alimentação e a dormir no emprego. Levantava de madrugada, às quatro
horas da manhã. Distribuía o pão para os fregueses que moravam mais longe e para os
estabelecimentos onde o pão era vendido. Esse era o meu trabalho diário, 
além da entrega do açúcar para os padeiros que assavam pão doce,
bolos, etc.}\looseness=-1

\textls[-5]{Durante esse período, apareceu um ladrão que roubava açúcar. Reparei que os sacos diminuíam de quantidade de um dia para o outro. Eles ficavam bem próximos da entrada, em um compartimento. Comecei a observar as pessoas que tinham chave e acesso à padaria, e podiam entrar de madrugada. Um rapaz, de quem eu já desconfiava, chegava antes dos outros. Um dia, acordei mais cedo e fiquei esperando escondido. Vi então que ele já estava ali fazia tempo, bem antes dos outros padeiros chegarem. Quando os outros apareceram, todos se juntaram para tomar mate. Ele não me viu, saí e procurei dentro das coisas dele. Debaixo do assento estavam pequenos sacos de açúcar. Despertei o dono e o levei até o local para mostrar que ele estava sendo roubado.}\looseness=-1

\textls[10]{Trabalhei na padaria por dois meses, tinha folga nos domingos durante apenas quatro
horas, e fazia o serviço que antes duas pessoas realizavam. Estava
muito cansado, mas a esposa do meu patrão ainda achou que era pouco e me 
tirou estas quatro horas de folga, para limpar a loja. Ela
queria me transformar em um escravo.}

\textls[-15]{Aos domingos, eu aproveitava esse curto espaço de tempo para encontrar 
meus \textit{irmãos de navio}}\looseness=-1\footnote{Irmãos de navio, em iídiche \textit{shif brider}, é a expressão utilizada para identificar imigrantes recém-chegados.} \textls[-15]{em um restaurante. Contei o que estava acontecendo, e resolvemos que eu não voltaria para a padaria naquele
momento, só à noite. Cheguei bem tarde e não levantei de madrugada para
trabalhar. O patrão foi me acordar, eu dei uma desculpa e ele
achou que eu estava doente. Então não assumi o trabalho que costumava fazer
pela manhã. Mais tarde, pedi as contas e disse que não trabalharia mais na
padaria. Ele me pagou o que quis, trinta pesos no mês. As duas pessoas que
trabalhavam lá antes de mim, fazendo o mesmo serviço, recebiam sessenta pesos cada um. Foi uma
exploração que cometeram, porque eu era \textit{gringo} e não tinha ninguém para me defender. Fui embora.}\looseness=-1

\chapter{Da capital ao interior}

Encontrei um novo problema: ter um lugar para dormir. 
Desempregado, não conseguiria pagar um quarto sozinho. O dinheiro recebido de meu antigo 
patrão estava guardado para comprar comida.

\textls[-18]{Na Corrientes, encontrei meus \textit{irmãos de navio} e contei o 
que acontecia comigo. Disseram-me que poderia dormir com eles, mas teria de
entrar tarde da noite, escondido, para o senhorio não perceber. Aceitei. Toda noite um deles me aguardava perto da porta, 
e fazia um sinal para entrar. Mas não havia espaço, as camas
eram de solteiro, e eles dormiam dois em cada uma. Só sobrava um lugar
no chão, que passou a ser o meu.}\looseness=-1

\textls[10]{Acordava bem cedo, e os donos da pensão nunca ficaram sabendo que estava
dormindo lá. Para comer, ia na cantina organizada 
para novos imigrantes que ainda não tinham trabalho. A alimentação 
era barata, não poderia pagar o preço de
80 centavos cobrado por outros restaurantes. Nessa cantina custava só 30
centavos. Quem não podia não pagava nada, comia de graça.}

\textls[-5]{A coletividade judaica tinha boas pessoas que colaboravam com os
\textit{griner}. Não os deixavam passar fome, e demonstravam solidariedade 
ao organizar cozinhas e cantinas populares. Além das refeições, era também mais
fácil conseguir um emprego nesses lugares: os empresários
procuravam por mão de obra não especializada, como era o meu caso.}\looseness=-1

Na sexta-feira à noite, apareceu um senhor que se aproximou da mesa onde
eu estava sentado comendo e perguntou: ``Quem de vocês precisa de trabalho?''

\textls[10]{Eu rapidamente dei um pulo e gritei: ``Eu preciso muito trabalhar'', embora não soubesse do que se tratava. E o senhor disse: ``Termine a sua refeição, eu vou esperar.''}

``Já acabei'', respondi, pois não conseguia comer de jeito algum. Fiquei
com receio que ele falasse com outra pessoa. ``Estou satisfeito, podemos
ir!''

\textls[10]{Tomamos o bonde e fomos até a casa dele. Não perguntei qual
seria o serviço, o importante era ter um trabalho. Quando
chegamos lá, ele me apresentou para a família e explicou sobre o
trabalho. Era o filho dele quem iria me ensinar.}

\textls[7]{Meu patrão tinha uma pequena fábrica de cigarros especiais, e
entregava encomendas para uma clientela selecionada: médicos,
advogados e pessoas ricas. Além de ajudar os trabalhadores na fabricação
dos cigarros, eu deveria entregar as encomendas a domicílio. E
limpar a loja, que ficava na frente da casa.}

\textls[10]{Passei a trabalhar lá, com direito a comida e moradia, e ainda
recebia gorjetas dos fregueses satisfeitos quando fazia entregas. Meu
patrão e a família gostavam de mim porque eu era leal, de toda
confiança. Eles também me levavam para todas as festas que eram convidados.}

\textls[10]{Trabalhei com eles durante quatro meses. Todos os domingos eu ia para a rua
Corrientes, a um restaurante cujo dono se chamava Charles. Lá encontrava meus 
amigos, os \textit{irmão de navio}, e ouvia notícias do
que estava acontecendo no mundo.}

\textls[-7]{Fiquei então sabendo sobre um trabalho no interior da Argentina, na
época da colheita. Ceifando se ganhava muito bem, embora fosse um
trabalho pesado, que ia desde o amanhecer até o pôr do sol. A vantagem é que
não se gasta nada do dinheiro, guarda-se tudo, pois não havia o que
comprar. Ao trabalhar por alguns meses, teria uma boa quantia.}\looseness=-1

\textls[-18]{Achei excelente, e não conseguia tirar a ideia da 
cabeça. Resolvi viajar imediatamente. Quando cheguei 
em casa à noite, contei aos meus patrões a resolução: ``Tenho a oportunidade de ganhar um bom dinheiro, e preciso disso para poder 
partir para América para me reunir aos meus irmãos.''}\looseness=-1

\textls[-5]{Meu patrão, sua mulher e os filhos me advertiram que este
trabalho era pesado demais, e as minhas mãos não estavam preparadas: eram
delicadas e eu não iria suportar, poderia até adoecer. Mas não
quis ouvir os conselhos, pensava só em ganhar bastante
dinheiro e me encontrar com meus irmãos na América. Sabia que
com dinheiro conseguiria obter um passaporte argentino --- como se já
vivesse ali há cinco anos ---, e não seria impedido 
de viajar. Preparei-me então para ir ao interior da
Argentina e trabalhar na colheita. O ano de 1922 já estava terminando.}\looseness=-1

\chapter{De charrete em Moisés Ville}

\textls[15]{Fui até a estação de trem para embarcar rumo a 
Moisés Ville,}\footnote{Moisés Ville é uma colônia agrícola fundada na província de
  Santa Fé, na Argentina, em 1889, com objetivo de proporcionar aos
  imigrantes judeus a oportunidade de se tornarem agricultores. Foi uma
  das colônias da \textit{Jewish Colonization Association}, instituição
  filantrópica fomentada pelo Barão Hirsch.} \textls[15]{uma cidade pequena onde
  havia muitos judeus. Cheguei pela manhã, encontrei alguns coches aguardando passageiros. Naquele tempo não havia
  automóveis para transporte, nem ônibus.}

\textls[10]{O cocheiro me perguntou para onde estava indo. Respondi que ia 
trabalhar na colheita. Ele me olhou e disse: ``Está chegando para a
colheita, mas ela já acabou. Já que veio até aqui, vou te levar até
Moisés Ville'', e me deixou na porta da cooperativa que os agricultores
judeus organizaram, e me apresentou a algumas pessoas.}

\textls[10]{Eles me fizeram muitas perguntas. Contei que vim trabalhar para ganhar dinheiro e poder me reunir com meus irmãos, pois 
na Argentina não tinha nenhum familiar. Escutaram-me com
atenção, e me falaram sobre um senhor com quem poderia trabalhar. Ele
viajava pelas aldeias e vendia utilidades domésticas em uma carroça
atrelada com cavalos, cheia de mercadorias: levava roupas para homens,
mulheres e crianças, além de bugigangas como pentes, espelhos, colares,
brincos, anéis, livros, discos e objetos de uso pessoal.}

Fomos até a casa dele e me apresentaram: passamos a trabalhar juntos.
Preparamos todas as mercadorias, e em alguns dias partimos em viagem. Vendemos bastante, e voltamos a Moisés Ville. Renovamos o estoque, e
viajamos novamente. Fiquei trabalhando dessa forma durante um ano. Ganhei um bom dinheiro e voltei para Buenos Aires. 

\textls[10]{Logo fui atrás do
senhor que providenciava os passaportes argentinos. Perguntei quanto ia me 
custar e quanto tempo levaria: me prometeu que em breve poderia viajar. 
Ele subia até o consulado americano e nós esperávamos embaixo. Junto com outras pessoas na mesma situação, nos víamos lá quase todos os dias. Quando descia, dizia
que precisávamos ter paciência, os passaportes ainda não tinham
ficado prontos para podermos ter os vistos.}

Ele disfarçou e ganhou tempo, sem dizer claramente que era impossível
receber os vistos no consulado. Quando descobrimos, pedimos
nosso dinheiro de volta. Ele não tinha mais o dinheiro, nem como devolvê-lo. Fugiu de Buenos
Aires para o Brasil, e lesou a todos nós. Perdemos tudo o que já tinha
sido pago, e eu fiquei sem dinheiro.

\chapter{Do trabalho aos palcos}

\textls[10]{Começava o ano de 1924. E eu com o mesmo antigo problema: procurar
trabalho. Novamente comprei um jornal iídiche, e achei um anúncio de uma
fábrica de correntes de ferro.}

\textls[-10]{Consegui o emprego, que consistia em tirar a ferrugem das
correntes para pintá-las em seguida. No primeiro dia, trabalhei até a hora
do almoço. Minhas roupas ficaram pretas de ferrugem, também a boca,
os olhos e o nariz. Comecei a cuspir ferrugem. Logo, vi que se
continuasse naquele trabalho iria acabar doente, e depois do
almoço não voltei mais. Comprei outro jornal e encontrei um anúncio de trabalho em uma loja.}\looseness=-1

\textls[-20]{Às cinco horas da manhã eu já estava na porta da loja esperando para abrirem. Era o primeiro da fila, atrás havia muitas pessoas. Quando o dono chegou, perguntou: ``Quem é o primeiro?'', e respondi: ``Eu''. Ele então me
convidou para entrar, e explicou sobre o serviço. Ele era folheiro, trabalhava com folhas de flandres, consertava
telhados, tinas, bacias e às vezes até fazia telhados novos. Precisava de
alguém para ajudar a carregar o material até o local de execução do trabalho. Perguntou se eu já tinha tomado o café da manhã, e eu disse que sim, embora não tivesse comido nada.}\looseness=-1

Peguei a caixa de ferramentas e fomos tomar o bonde, para chegar ao 
local do trabalho. De repente ele se deu conta de que se esqueceu 
de levar a escada, e me perguntou se eu acertaria o caminho até 
chegar à loja: era preciso buscar a escada, caso contrário não 
seria possível instalar as calhas. Falei que iria até a loja e 
voltei de bonde. Mas, na volta, não me deixaram subir no bonde com a escada.

\textls[10]{Andei durante algumas horas com a escada no ombro até chegar ao local
onde meu patrão estava trabalhando. Ele consertou as calhas e eu pensei
que teria de levar a escada de volta, mas, graças a Deus, a escada deveria
permanecer lá mesmo.}

\textls[10]{Voltamos de bonde até a casa dele. Já estava escuro, me lavei e fui
convidado para jantar. A mesa já estava servida, com diversos
pratos como peixe cozido, chá e leite.}

\textls[15]{Não achei o aspecto do peixe muito agradável, parecia que tinha sido 
mexido demais. Então eu disse que não estava com fome, embora tivesse
passado o dia inteiro sem comer nada. Disse que ia buscar os meus
pertences. Fui embora e não voltei mais.}

\textls[5]{Precisei outra vez procurar trabalho, comprei o jornal e vi um anúncio para vendedor em uma loja de ternos para pronta entrega. Eu fui, me aceitaram e eu fiquei. Meu trabalho era varrer a loja e
entregar os ternos em outras lojas, atendendo os pedidos.
Trabalhei ali alguns meses. Como eu era \textit{griner}, os patrões achavam
que deviam me pagar menos, o que não era suficiente para o aluguel
de um quarto e alimentação. Como eu ainda tinha algum dinheiro de
Moisés Ville, fui me sustentando. Achava que em algum momento receberia um aumento, e
a situação melhoraria.}

\textls[5]{Depois de alguns meses, como não recebi o aumento, resolvi procurar
 outro serviço. Encontrei em uma mercearia, onde poderia comer e dormir. Meu trabalho era deixar a loja limpa e entregar as compras nas
 casas das freguesas. Durante o dia, por cerca de duas horas, carregava ainda duas
 latas de óleo de 20 litros cada, para venda a domicílio. Quando
 fechavam a loja, colocavam uma cama para eu dormir após um dia pesado
 de trabalho. Neste serviço fiquei alguns meses.} 

\textls[10]{Nunca perguntava
quanto iam me pagar ou quando começava o trabalho: e neste aqui também não
perguntei. Às vezes, queriam que eu trabalhasse somente
por cama e comida.}

Eu estava sozinho, sem ninguém que pudesse interceder. Quando disse
para o patrão que queria receber mais, ele se virou para esposa e
disse: ``Olha só esse \textit{griner}, ele já saciou sua fome e quer
ganhar mais! Você pode ir embora, porque não faltam \textit{griners} que
queiram trabalhar''. Então eu fui embora.

Novamente me dirigi à rua Corrientes e me encontrei com os meus \textit{irmãos
de navio}. Contei para eles que estava outra vez sem trabalho, e um
deles me disse que havia uma firma suíça aceitando empregados. 
Caso me perguntassem se tinha experiência neste
tipo de serviço, eu deveria falar que sim, do contrário não me
aceitariam. Disseram-me existir várias firmas que faziam o mesmo
tipo de trabalho.

Fui até aquela firma, e quando me perguntaram se já tinha
trabalhado nesse tipo de serviço, respondi que sim. Fui aceito. O
trabalho consistia na construção de casas novas, e eu devia ajudar o mestre
de obras na instalação das calhas de folhas de flandres nas casas: ele já trabalhava nisso há algum tempo. 
Apresentei-me, dizendo
que tinha sido contratado para ajudá-lo. O mestre de obras era alemão, e eu
podia conversar e me entender com ele em iídiche. Mandou-me fazer a base
para a calha, peguei um balde, o alicate com arame e pregos e subi para
fixar o calço na base. Quando olhei para baixo e vi que as pessoas
passando na rua pareciam formigas, comecei a tremer de medo. Ele
reparou nisso e me mandou descer. Então confessei a verdade, que
nunca tinha trabalhado neste tipo de serviço, mas que precisava
trabalhar para ganhar meu sustento, porque na Argentina não
tinha nenhum parente.

\textls[-15]{O mestre de obras disse: ``Aqui estamos só nós dois e o dono não vai
saber quem está instalando, eu o farei até você se acostumar com esse
serviço.'' E assim acabei ficando lá, pagavam bem. Perdi o medo e
com o tempo fiquei esperto como um gato, subia nos telhados, fazia de
tudo. Trabalhei assim alguns meses, até que começou a faltar trabalho
para a firma e tiveram que começar a demitir funcionários --- e eu também fui demitido.}\looseness=-1

Novamente desempregado, procurei um emprego e achei numa loja de judeus
sefarditas. Era um armarinho. Eu empacotava a
mercadoria e limpava a loja, fazia tudo que era necessário. Morava com
uma família, onde também morava um rapaz da minha idade. Ele
participava de um curso de arte dramática, que comecei a frequentar com ele.

\textls[5]{Nessa época, foi escrita uma peça de teatro para ser apresentada 
com artistas judeus, em iídiche. Mas havia atores que queriam cortar muitas cenas da peça, que se chamava \textit{Ibergus} ou \textit{A transmutação}, de Leib Malach.}\footnote{A rede
  internacional de tráfico de mulheres judias, Zwi Migdal, possuía uma
  ampla rede na América do Sul, naquele período. A convivência entre
  homens e mulheres ligados ao Zwi Migdal e setores mais formais das
  comunidades judaicas era tensa e dissociada. É interessante notar que
  a obra \textit{Ibergus}, de Leib Malach, trata desse tema.} \textls[5]{O público que ocupava os melhores lugares eram cafetões, traficantes de mulheres (escravas brancas), donas de casas de tolerância e mulheres de vida fácil. Houve um confronto
  entre o autor, os artistas e a imprensa iídiche, que começou a clamar
  estarem fazendo uma afronta ao teatro iídiche. Exigiam um \textit{teatro puro}, onde aquelas pessoas indesejáveis não pudessem entrar.}

Dirigiram-se ao curso de arte dramática, e pediram aos alunos que 
não apresentassem a peça. Mas o grande escritor e jornalista Yankel
Batochansky estava em luta conosco. Começamos a ensaiar e um mês depois já estávamos apresentando a peça
em um grande teatro. Foi um enorme sucesso, e eu também participei.

\textls[10]{Os artistas começaram a brigar entre si, houve uma ruptura entre os grupos. Um dos 
elencos de atores juntou-se a nós, entre eles o ator Zaslavsy e sua esposa, assim como Naumof, Klinguer,
Warschawsky e outro atores e atrizes cujos nomes não me recordo.
Convidaram os companheiros do curso dramático para trabalhar com eles na
peça, e eu também entrei para representar com esses artistas famosos. Trabalhei com eles durante seis meses.}

\textls[10]{Chegaram então atores e atrizes da América do Norte, e
representamos juntos. Não lembro seus nomes, mas vou mencionar as peças nas
quais participamos com este artistas: \textit{Der Ibergus} e \textit{Drai un Draissic ior Colonizatzie}, ambas de Leib Malach, \textit{Tzivshn tog un nacht}, de Leon Alpern, \textit{Graf Patotzki}, \textit{Schver tzu zain a iid}, \textit{Der Rumenische Chassene}, e
muitas outras peças.}\footnote{Em sequência, as traduções dos títulos das peças são: \textit{A transmutação}, \textit{Trinta e três anos de colonização}, \textit{Entre o dia e a noite}, \textit{O Conde Patotzki}, \textit{É difícil ser um judeu} e \textit{O casamento romeno}.}

\textls[10]{Na mesma época, me sugeriram que através dos atores americanos 
eu poderia conseguir um visto de viagem para a América --- eles poderiam me levar, como ator. Mas não foi possível, e
enquanto estive com eles gastei todo dinheiro que tinha poupado
durante meio ano. Tudo o que tinha conseguido desde que cheguei à Argentina.}

\textls[10]{Fiquei com pouco dinheiro. Disseram que poderia viajar
para me encontrar com meus irmãos na América passando pelo Uruguai.
Que seria fácil, não precisaria de passaporte, apenas da
carteira de identidade argentina. Então resolvi ir: tomei um navio e fui
para Montevidéu. Já estávamos no ano de 1926.}

\chapter{Primeira parada rumo ao norte}
% \addcontentsline{toc}{chapter}{Primeira parada rumo ao Norte}
% \markboth{Primeira parada\ldots}{}

\textls[10]{Desembarquei em Montevidéu pela manhã, e no porto perguntei 
onde moravam os judeus da cidade. Disseram-me que não muito
longe do cais havia um salão de barbearia, cujo dono era judeu.}

\textls[2]{Fui até lá. Contei que acabara de chegar de navio da Argentina, mostrei
meus documentos e perguntei onde poderia morar. Ele disse que possuía
uma pensão e poderia me arranjar um lugar na casa dele, em um
quarto junto como outros rapazes, e que iria me custar bem pouco. Eu
 também poderia comer na pensão junto com os outros hóspedes. Mandou uma
pessoa me acompanhar e mostrar a casa. Acabei morando lá. De noite,
quando chegou em casa, conversei com ele e contei tudo que estava
acontecendo comigo: que chegara a Montevidéu para tentar
viajar para América e me reunir com meus irmãos. Ao mesmo tempo,
perguntei como conseguiria algum trabalho.}

\textls[10]{Ele disse que conhecia quem pudesse me arranjar um trabalho, e me deu seu
endereço. No dia seguinte, fui até lá: o homem me deu uma carta para um
conhecido seu da Companhia de Bondes. Eu fui, e consegui o emprego.}

\textls[10]{Recebi um quepe com um número em metal. Desse modo eu poderia viajar de
bonde sem pagar, era só colocar o quepe na cabeça enquanto viajava no
bonde até a Companhia dos Bondes, enquanto aprendia. Demorou um mês 
até eu prestar o exame e me admitirem --- para começar a trabalhar e ganhar 
meu salário. Passado o exame, comecei o trabalho 
como condutor de bondes, e cobrava também o dinheiro das passagens dos usuários.}

Almoçava na pensão onde morava, e à noite jantava depois do serviço. Além dos hóspedes, 
vinham pessoas de fora comer na pensão, e o dono sempre
dizia que sua esposa estava para chegar. Um belo dia ela chegou, e os
dois vinham comer na pensão. Na época, foi celebrado lá um casamento: a irmã
da dona da pensão casou-se com um rapaz que morava lá. Eu participei da
festa e até convidei uma moça para dançar. Achei estranha a maneira dela 
rir e não gostei de como dançava e se comportava. Mas não podia
falar mal dela, pois não a conhecia bem. Quem sabe era o seu jeito.
Depois de um curto período na pensão, o casal partiu para outra cidade,
onde ele disse possuir uma loja de roupas usadas, que estava até então 
fechada. Acabamos nos esquecendo deles, mas voltarei mais adiante a este
casal quando falar sobre o Brasil --- pois fui, em seguida, de navio para o
Rio de Janeiro.

\textls[-10]{Trabalhei em Montevidéu até o início de 1928. Consegui guardar um pouco
de dinheiro para poder viajar ao encontro de meus irmãos e irmãs.}\looseness=-1

\chapter{Rio de Janeiro e seus caminhos}
% \addcontentsline{toc}{chapter}{Rio de Janeiro e seus caminhos}
% \markboth{Rio de Janeiro\ldots}{}

Quando desembarquei no Rio de Janeiro, dormi em uma pensão. No
jantar, a mesa foi posta e serviram feijão preto. Fiquei observando
pois nunca tinha visto isso, e não consegui comer. Saí na rua, entrei
em um bar e pedi café com leite e pão com manteiga, e voltei para a
pensão para dormir. De manhã, fui procurar onde ficavam os judeus. Andei
e, de repente, em plena Praça Onze, encontrei muitos judeus.
Perguntei onde poderia achar um quarto para morar, e me
mostraram uma pensão onde uma senhora viúva alugava quartos e fornecia
refeições.

Dirigi-me para lá e conversei com a dona. Ela me disse que no
momento não tinha quarto vago, mas se eu aguardasse um
pouco poderia dormir na sala, onde dormia outro rapaz, e havia espaço para mais um. 
Eu aceitei as condições oferecidas, porque
queria ficar entre judeus. Voltei para a pensão onde dormira antes, peguei minhas coisas e levei para o novo local. À noite, foram colocadas duas camas, uma para o
rapaz que eu ainda não conhecia e outra para mim.

Já estava dormindo quando algo começou a me picar, a morder, acho que o
rapaz também sentiu a mesma coisa. Só não conversamos afinal porque não nos conhecíamos. Um esperava que o outro tomasse a iniciativa de
falar, e enquanto isso os percevejos estavam fazendo a festa. Se tornou
difícil de suportar, até que um de nós acabou falando e reclamando,
não lembro quem falou primeiro. Não sabíamos onde acender a
luz. Pulamos das camas, no escuro, procurando pela porta.
Achamos, e ela dava para um quintal, onde vimos uma mesa.
Resolvemos dormir em cima dela. Retiramos os travesseiros e lençóis,
sacudimos bem até não sobrar nenhum percevejo, deitamos e dormimos muito
bem.

\textls[-10]{Quando amanheceu, acordamos e rimos muito, mal podíamos falar. Foi tal o
acesso de riso que acabamos acordando todos na pensão, que vieram
correndo para ver o que estava acontecendo. Por que estávamos rindo tão
alto, será que tínhamos ficado doidos? Quando se aproximaram, viram
como ambos ficamos pretos e também começaram a rir junto conosco.
Nós acabamos dormindo bem perto de onde passavam os trens, e as
locomotivas soltavam uma fumaça preta. O vento jogava e espalhava a
fumaça, que caía justo em cima de nós, então ficamos bem sujos.}\looseness=-1

\textls[-15]{Após tomar um bom banho, fui até a Praça Onze me encontrar com os judeus
e ver como podia tentar resolver os meus problemas. Conversando com eles, surgiu
uma ideia: quem sabe viajar para o México seria uma forma mais prática de chegar mais perto dos
meus irmãos? Minhas duas irmãs já estavam na América e também o meu
irmão mais velho, éramos cinco irmãos e duas irmãs, e eu era o caçula.
Dirigi-me então ao consulado mexicano e pedi o visto. Disseram-me para
trazer o meu passaporte, assim poderia partir para o
México --- mas eu havia perdido meu velho passaporte durante tantas
viagens de um lugar para o outro.}\looseness=-1

\textls[-2]{Fui até o consulado da Polônia, levando a carteira de identidade
argentina, com a qual saí e cheguei no Uruguai, e mais tarde vim para o
Brasil. Não houve necessidade de passaporte: não exigiam nada mais,
simplesmente me deixaram partir. O cônsul polonês, entretanto, disse que
não podia dar um passaporte sem antes consultar o governo polonês sobre
mim, depois disso ele poderia fornecer o passaporte.
Perguntei quanto tempo levaria até receber uma resposta, e ele disse
que mais ou menos uns seis meses. Eu já imaginava que quando ele tivesse
a resposta, me recusaria o passaporte porque saí da Polônia na época que
fui convocado para servir o exército e não voltei mais.}

% acabei
% ficando, e não pude partir ao encontro dos meus irmãos e irmãs na
% América.

\textls[-5]{Voltei novamente até a Praça Onze, onde conheci mais gente. Entre
eles, encontrei um homem que já tinha conhecido no Uruguai, durante o seu
casamento, na pensão onde eu estava morando. No capítulo anterior, 
escrevi sobre ele e a esposa, como ambos desapareceram, e ninguém
ficou sabendo para onde tinham ido. Ele me reconheceu, e eu perguntei o
que ele estava fazendo no Rio de Janeiro. Disse que morava numa
pequena cidade próxima do Rio e que tinha uma loja de roupas. Possuía uma 
equipe de vendedores que viajam à cavalo pelas cidades do
interior, vendendo à vista e à prazo. Perguntei se poderia trabalhar para
ele, disse que poderia ser. Ficamos de nos encontrar no dia seguinte
para conversar, marcamos o horário e o lugar, mas ele não apareceu e
nunca mais o vi. Pensei que, com certeza, havia precisado partir de
repente. Ou que, talvez, tivesse acontecido alguma desgraça, mas
aqueles que o conheciam e sabiam de seus negócios não davam nenhuma
informação sobre o seu paradeiro. Como me viram o dia inteiro em sua
companhia, acharam que eu fazia parte da turma dele, e me ignoraram.}\looseness=-1

\textls[-15]{Fui novamente até a Praça Onze, e pedi uma indicação de
emprego. Com um endereço que me deram, tomei um ônibus e, quando
sentei, reconheci a mulher do tal sujeito acompanhada de outra
mulher no ônibus. Sentei bem atrás, esperando que descessem 
para que pudesse segui-las disfarçado. Elas entraram em um café
e eu também. Fiquei sentado num canto, pedi um café. Percebi então
quem eram aquelas pessoas: ele, um cafetão, e ela, uma \textit{tia}, ou dona de uma \textit{casa de tolerância}, ambos traficantes de escravas
brancas para prostituição. Podem imaginar onde eu podia ter ido parar 
sem saber! Foi então que entendi por que os outros judeus me olhavam com
antipatia, mas Deus sempre me conduziu no bom caminho, e continua me
protegendo, para que eu não fique envergonhado.}\looseness=-1

\textls[10]{Consegui arranjar um trabalho de mascate, com um judeu de nome Israel
Soifer, que morava no Méier, um bairro do Rio de Janeiro. Ele morava
com um irmão, cunhada e filhos, e ainda uma irmã que chegara há pouco
de Britchev, Bessarábia. Convidaram-me para morar com eles, e era muito
agradável o ambiente familiar. Todos apreciavam o meu comportamento,
até a cunhada de Israel --- que queria que eu casasse com a irmã dela. A moça
era muito bonita, mas eu não estava com pressa, a minha ideia continuava
a mesma: chegar até a América, para ficar com meus irmãos e irmãs.}

\chapter{Tirando a sorte}

\textls[10]{Tomei um navio e cheguei até o Recife. Durante dois dias, o navio permaneceu no porto para descarregar. Desci e fui levar um pacote para os 
Kutner, a família com quem morei no Rio de Janeiro havia me pedido para entregar. 
Achei o endereço e entreguei o pacote, e fiquei até voltar a embarcar. Contei para eles que estava indo para América encontrar meus irmãos e irmãs.}

\textls[-3]{Voltei ao navio e viajei para Belém do Pará. Assim que cheguei, fui para
uma pensão, e comecei a indagar sobre os petroleiros. Informaram-me que estes navios só apareciam a cada seis meses, e logo vi que
não poderia prosseguir com a minha viagem muito brevemente.}\looseness=-1

\textls[10]{Voltei para a pensão e perguntei se havia judeus na cidade. Naquela
época no Pará só havia um pouco de judeus sefarditas, mas o vizinho do meu lado na pensão 
era um judeu húngaro. Travei amizade com ele e
contei a minha história, de quanto já tinha viajado para tentar me
reunir com meus irmãos. Mas já não sabia mais o que faria, e precisava novamente
arranjar um trabalho.}

\textls[10]{Ele me perguntou se tinha um pouco de dinheiro. Afirmei que sim,
então me fez uma proposta de abrir um negócio, como sócios. Ele tinha uma
pequena fábrica e gostaria de ampliar, mas não tinha capital para
investir. Se eu entrasse com o dinheiro, nós teríamos a chance de obter
bons lucros. Aceitei a sociedade e juntos fomos comprar os materiais
necessários, e começamos a trabalhar dentro de nossos quartos da pensão.
Conseguimos fabricar uma porção de quadros. Ele saía para vender e eu
permanecia no quarto trabalhando, preparando outros. Passaram vários
dias, ele não me prestava contas, e nem dinheiro eu via, assim me
convenci que tinha entrado numa fria e que ele estava trapaceando. Peguei
o material restante e levei para o meu quarto, e saí procurando alguns
judeus para contar o que estava acontecendo comigo. Pedi que
fossem falar com ele para devolver o meu dinheiro e fazer o acerto de
contas entre nós.}

\textls[10]{Quando voltei à pensão, encontrei-o em companhia de dois investigadores. 
Ele me apontou assim que entrei, me disseram que eu devia
acompanhá-los até a delegacia de polícia, porque estava sendo acusado de
ter roubado quadros do quarto dele. Fiquei sem reagir e pensei comigo
mesmo: \textit{Que má sorte, tudo tem que me acontecer! Só faltava eu ser
acusado de ladrão!} Mas é assim mesmo. Quando a pessoa está só no
mundo, sem família, precisa passar por tudo isso. E lá eu não conhecia
ninguém. Depois de tudo que já tinha passado na vida, só porque queria
ser sempre honesto e manter o comportamento decente que herdei dos meus
pais, fazer somente o bem, preservar o bom nome da família em
qualquer situação, teria que ser firme e forte.}

\textls[-10]{Fui para a delegacia. O delegado leu para mim o que o meu sócio
declarou que eu havia roubado. Ouvi tudo atenciosamente e depois contei
para o delegado o que na realidade havia acontecido entre mim e aquele
que me acusou. Em primeiro lugar, não fui eu quem o havia
roubado, e sim ele. Montamos uma sociedade e tínhamos direitos iguais, 
entrávamos um no quarto do outro. Ambos tínhamos
as chaves, porque o material que usávamos no nosso trabalho
ficava armazenado nos dois quartos. Eu não tinha roubado nada, e pretendia
chamar algumas pessoas para falar sobre este assunto, pois era ele quem não
me prestava contas sobre as vendas. Eu precisava do dinheiro investido de volta, não queria
mais ser sócio dele. O delegado viu que estava contando a verdade e me
liberou.}\looseness=-1

\textls[5]{Meu sócio permaneceu lá. O que eles conversaram não sei, mas fui ao
encontro dos meus amigos judeus e contei o que aconteceu. Um deles
logo telefonou para o delegado e disse que algumas pessoas queriam ir
conversar com ele. O delegado respondeu que podiam ir, fui junto e 
meus amigos confirmaram tudo o que eu já havia contado antes, e que eu
pedi a intervenção deles para resolverem com justiça a questão. O
delegado quis abrir um processo contra meu antigo sócio e prendê-lo,
mas eu não deixei. Só queria que devolvesse o meu dinheiro. Não sou
vingativo, apenas uma pessoa de bem. Foi o que meu pai me ensinou: não
provocar o sofrimento de ninguém. O delegado ouviu meu pedido e,
dirigindo-se ao meu acusador frustrado, disse: ``Você devia se
envergonhar, é mais velho que Jaime, tem idade para ser seu pai! Que
belo exemplo é a compostura dele perdoando-o, que educação recebeu!
Devolva o dinheiro que ele investiu e saia da minha frente e do estado
do Pará, nunca mais apareça aqui, diante dos meus olhos.'' Assim, 
consegui receber o meu dinheiro de volta.}

Enquanto estou escrevendo a minha história de vida, sinto muito orgulho
no coração. Escrevendo sobre este caso, me sinto tão magoado e
injustiçado, com vontade de chorar. Imaginem, tentaram me transformar em
um ladrão! E por que tinha que acontecer logo comigo?

\textls[10]{Comecei novamente a procurar emprego. Achei
que Deus não tinha me abandonado, tinha meus méritos, e nunca fiz 
mal a ninguém! }

\textls[5]{Na época, morava
um senhor que tinha um circo na cidade na mesma pensão onde eu estava. Ele soube do que
aconteceu comigo, e conversando com ele contei que procurava um trabalho
porque o meu dinheiro estava acabando.}

Ele me ouviu com atenção e disse que, se quisesse, teria um serviço para
mim.

``Claro'', respondi, ``se o senhor tem um trabalho para mim, seria muito
bom''. ``Venha ao meu quarto'', disse o senhor. Mostrou-me uma caixa
revestida de pano preto, era um realejo. Explicou-me que era uma máquina
de tirar a sorte. Havia uma manivela e duas rodas dentro, precisava 
rodar com a manivela para que aparecesse um papel branco dobrado dentro
dela escrito \textit{sorte}, junto com algumas palavras boas.

\textls[-5]{Aceitei o emprego e concordei com tudo. Ele me ensinou como devia fazer
e disse que o lucro seria dividido entre nós. Peguei logo a caixa e
fui para a rua, trabalhei algumas horas com o realejo e consegui quarenta
mil réis. Era muito dinheiro, e fiquei trabalhando bem contente.
Ele me perguntou se eu gostaria de trabalhar algumas horas à noite, pois
tinha outro trabalho para mim. Respondi que seria bom, trabalhando mais,
conseguiria ganhar mais, então fui trabalhar no parque de diversões ao
lado do circo, com a roleta. O trato era de que eu ganharia 30\% do valor. Viu que eu
não era preguiçoso, e só confiava em mim.}\looseness=-1

\textls[5]{Assim, começamos a trabalhar na cidade de Belém do Pará. Depois de um
mês, viajamos pelas pequenas cidades do interior, de uma para
outra, até chegarmos a Manaus, no Amazonas. Comecei logo a trabalhar na
rua, mas de repente apareceram dois fiscais e perguntaram se eu tinha uma
licença para trabalhar ali. Evidentemente não tinha, e eles me
levaram para a delegacia de polícia e me prenderam. E me deixaram ali,
porque o delegado não estava. Era domingo e ele só permanecia algumas
horas do dia, e ainda era cedo. Fiquei aguardando bastante tempo na
sala, comecei a sentir fome, pedi ao guarda que me acompanhasse até o
restaurante, eu pagaria sua refeição. Ele aceitou. Depois de comer
bem, voltamos à delegacia, mas o delegado estava demorando a chegar,
então comecei a tirar a sorte para os guardas que ali se encontravam.
Quando finalmente o delegado chegou, eles contaram tudo a ele, e eu
tirei a sua sorte. Ele começou a rir, pois a mensagem que eu tirei lhe
agradou muito, então ele disse que era uma atividade que não exigia
licença e mandou me soltar.}

\textls[-5]{Só que eu saí em liberdade, mas o realejo permaneceu apreendido. No dia
seguinte devia ir até a prefeitura. Fui na manhã seguinte, como tinha
prometido ao delegado, e falei com um funcionário. Contei tudo que tinha
acontecido e ele me ouviu com atenção, então pedi que telefonasse para a
delegacia e liberasse a ordem para me devolverem a minha caixa, assim poderia
ir buscá-la e mostrar como funcionava. Ele ligou logo e deu a ordem, e
disse que podia ir buscar o realejo e voltar. Claro que eu fiz tudo
direito e também para ele tirei a sorte. Isso provocou nele muitas
risadas, pois lendo o bilhete viu que estava escrito que ele era uma
pessoa de instrução elevada e um futuro brilhante o aguardava.}\looseness=-1

\textls[-10]{Chamou os outros funcionários: eu tirei a sorte deles, e mandou
que todos me pagassem. Embora não houvesse licença para a minha
atividade, fez questão de me dar uma que custaria um pouco, mas
impediria que algum outro fiscal me incomodasse na rua. Depois, pediu que
fosse à casa dele para tirar a sorte de toda a sua família, esposa e
filhas. Elas gostaram e me recomendaram a outros parentes. Consegui, assim,
trabalhar bastante tempo porque no realejo tinham muitos bilhetes que
serviam tanto para homens como para mulheres, escritos de 26 
modos diferentes. E agradava a todos. Eu trabalhava
todos os dias com o realejo e à noite no parque, mas no parque não
pagavam em dinheiro, e sim com prendas.}\looseness=-1

\textls[7]{Depois de um tempo, o dono do circo resolveu viajar para o Peru e eu precisei 
ir até uma delegacia fazer um passaporte, porque em
Manaus não tinha consulado polonês. Mas não pude viajar, fiquei
doente. Fui acometido de malária, ou \textit{sezão}. A doença
provocava altas temperaturas, sentia calor e depois frio de tremer.
Todos os dias mais ou menos às 11 horas eu tinha muita febre, logo em
seguida um frio} 

\pagebreak

\noindent{}\textls[7]{terrível --- e tinha que ir ao hospital, onde recebia
medicação. Não adiantou muito, eu não melhorava. Então resolvi não
acompanhar o circo, todos partiram e eu fiquei novamente só.}

\chapter{Um passo atrás}
%A malária e o percurso pelas capitais do nordeste.

\textls[-15]{Fui até a marcenaria e mandei fazer uma caixa parecida com aquela do
realejo, continuei trabalhando sozinho tirando a sorte. Fui até o
cemitério, comprei uma caveira --- dizendo que precisava dela para estudos  ---,
lavei bem com água sanitária, pintei e coloquei luz nos orifícios dos
olhos, usando duas pequenas lâmpadas à bateria. Fui a uma tipografia e
fiz os bilhetes, como os que haviam na caixa antiga, e comecei a trabalhar. Eu
já era dono do meu negócio, e não precisava dividir os lucros com
ninguém.}\looseness=-1

\textls[10]{Quando me despedi do pessoal que trabalhava no circo, lhes disse que
estava doente e voltaria para o Pará. Não contei nada sobre meu plano
para o patrão. Embarquei em um navio e fui até Belém do Pará e comecei a
trabalhar novamente, mas desta vez como dono do negócio. Não fui tão bem 
como antes, tentei durante algum tempo. Tomei então um navio para o 
Ceará, e depois para Natal no Rio Grande do Norte, enquanto continuava
trabalhando na rua tirando a sorte. Desta maneira consegui me
sustentar.}

\textls[7]{Um dia, se aproximou um homem e me perguntou se eu era judeu, respondi
 que sim. Ele ficou contente, e me contou que também era judeu. Acabou me
 convidando para ir à casa dele, deixou comigo o endereço e à noite fui até lá.
 Contei toda minha história, o que estava acontecendo em minha vida e
 também meu maior desejo, que era me reunir com meus irmãos e irmãs na América
 do Norte. Mas que agora tinha ficado doente com malária.} 

Esse senhor se chamava Chaim Hurvitz, e eu ia todas as noites à casa dele. Até que uma vez, quando cheguei, encontrei ele e a esposa vestidos para sair. Disse que não havia problema, voltaria para a pensão e viria outra noite. Ele me disse: ``Não, você virá conosco''. Perguntei para onde iam, mas a resposta foi: ``Onde nós vamos, você vai junto''. Então fomos andando até a residência da família Palatnik, onde acontecia um banquete em homenagem a um \textit{enviado} de Eretz Israel.\footnote{Antes da criação do Estado de Israel, em 1948, era comum o uso da expressão \textit{Eretz Israel} ou \textit{Terra de Israel} para designar a Palestina sob mandato britânico. A recepção do \textit{enviado} reunia a população judaica da cidade, uma das práticas de popularização do movimento sionista.} 

\textls[5]{Não me lembro o nome dele, mas sei que era um capitão de navio. Todos os judeus da cidade, com suas famílias, compareceram à recepção. Meu amigo Chaim me apresentou para toda a coletividade e especialmente ao dono da casa, Tobias Palatnik. Contou para ele quem eu era e por quantos países já havia passado: Argentina, Uruguai e Brasil --- nesse último, pelos muitos estados. O senhor Tobias me levou até a biblioteca e mostrou uma foto em que estava com o presidente do Brasil, naquela época Washington Luís. E disse: ``Como você viaja muito, pode contar às pessoas sobre esta foto e vão acreditar em você, pois muitos não acreditam. Mas você é testemunha que é verdade''.}

\textls[10]{Fiquei em Natal durante algum tempo e depois parti, desta vez de
trem e ônibus. Acabei perdendo a minha bagagem: não sei se roubaram ou
se caiu do ônibus em algum momento. Era costume colocar as malas
em cima do ônibus. Fiquei apenas com a roupa do corpo, até meu 
dinheiro estava dentro da bagagem. Minha sorte foi
que o realejo não sumiu. Continuei viajando de um lugar para outro,
tirando a sorte com os bilhetes. Cheguei então a uma cidade chamada
Campina Grande, e depois a João Pessoa, capital do estado da Paraíba.}

\textls[5]{Trabalhava com o realejo em toda parte, e me sustentei tirando a sorte. Permanecia 
 durante alguns dias na cidade e depois seguia para outras. Mas a doença não passava, e ficava muitas vezes de cama e com febre. Cheguei a uma cidade pequena, Itabaiana, e lá fiquei também com febre e
 acamado. Vinha o frio, não podia me levantar, a crise me derrubava todos os dias até cerca de meio-dia.} 

De repente, ouvi tocar um violino, eram
as melodias iídiche muito conhecidas: \textit{Habrivele Der Mamen},\footnote{Em tradução, \textit{Uma carta para minha mãe}.} era uma delas. Imaginem como ficou meu coração ao ouvir as músicas em iídiche após tanto tempo. Imaginei logo que ali devia haver judeus.

\textls[10]{Quando melhorei um pouco, saí para trabalhar e encontrei um senhor judeu
chamado Yossel Serigorski. Ele vendia a crédito, e vinha durante alguns
dias até ali para receber o dinheiro dos fregueses. Morava em Recife, capital de Pernambuco.}

Depois disso viajei para outra cidade, Timbaúba, e me hospedei em
uma pensão. Mas tive novamente uma crise de malária e o dono, vendo que passava mal, disse para ir embora porque não queria doentes
em sua pensão. Enfraquecido e muito doente, precisei sair e viajar até
chegar a Olinda, cidade vizinha de Recife, no estado de Pernambuco.

\chapter{O casamento em Recife}

\textls[-15]{Em Olinda, recomecei meu trabalho na rua. Entrei em uma
lanchonete para tomar um café, e vi um senhor judeu sentado em uma mesa
de canto. Ele lia um jornal em iídiche quando me aproximei e me
apresentei. O nome dele era Shimen Massur. Perguntei se
no Recife existia alguma pensão de judeus: ele disse que sim e
me deu o endereço, o proprietário se chamava Weber. Viajei até lá e
consegui um quarto, as refeições também eram servidas no local.
Depois de resolver a parte da moradia, fui procurar pela comunidade judaica. Ainda
guardava o endereço da família Kutner.}\looseness=-1

\textls[-15]{Dirigi-me até a casa deles e bati na porta, a senhora Kutner abriu e
perguntou o que eu desejava. Respondi que meu nome era Chaim e que já estivera ali algum tempo
atrás: naquela ocasião ia a Belém do Pará, para tentar um trabalho em navio petroleiro. Imediatamente ela se
lembrou e me mandou entrar, e perguntou por que eu ainda não tinha viajado. 
Contei tudo que me aconteceu depois de ter entregado a eles a encomenda do Rio de Janeiro. Ela me achou muito magro e pálido, por causa da malária.}\looseness=-1

Tinha se passado um ano desde que estivera com eles, já estávamos no ano de
1929.

\textls[-10]{Quando o marido dela chegou do trabalho --- ele era \textit{klienteltchik} ---,}\looseness=-1\footnote{Em tradução, \textit{mascate}.} 
\textls[-10]{fiquei conversando com o casal, jantei com eles e
voltei à pensão só para dormir. De manhã, recomecei a trabalhar na rua, e
no final da tarde ia até a casa deles todos os dias, jogávamos cartas e
conversávamos. Eu pedia conselhos, falávamos de minha situação e de como
a enfermidade tinha atrapalhado meus planos de viagem rumo à América do Norte.}\looseness=-1

\textls[-20]{Todos os dias, ia até a Praça Maciel Pinheiro, onde me encontrava com
alguns judeus. Eles perceberam que estava doente. Um deles,
chamado Yankel Lederman, me disse para ir até lá no dia seguinte cedo: 
ele me levaria até o hospital, para que o médico me examinasse e passasse uma receita.}\looseness=-1

\textls[-20]{No dia seguinte, fui levado bem cedo ao hospital, e o médico me
receitou \textit{quinino}. Comprei o remédio, paguei cinco mil réis, e assim que
tomei, comecei a melhorar. Quando estava terminando o remédio,
repeti a receita e a malária sumiu, graças a Deus estava recuperando 
minha saúde.}\looseness=-1

\textls[10]{Continuei indo à casa de meus amigos Kutner todos os dias depois do
trabalho. Um dia a senhora Kutner me disse: ``O que adianta você viajar
tanto se não dá nenhum resultado positivo? Você até ficou doente
nas tentativas de se reunir com seus irmãos e irmãs, e continua
sozinho. Você é um rapaz tão jeitoso, será que não chegou a hora de interromper 
suas viagens, pensar em casar e ter uma família? Ao invés de ficar
aqui conosco toda noite, não quer conhecer uma moça que nossos amigos
receberam, uma irmã que chegou da Europa? Ela é muito bonita, e a família
muito fina. O que acha dessa ideia?''}

\textls[-5]{A senhora Kutner continuou: ``Nós gostamos muito de seu comportamento,
gentil e decente, e seria do nosso agrado que conhecesse essa moça.
Quem sabe vocês acabam casando e estabelecendo uma vida aqui''. Então
eu respondi: ``Como posso pensar em casamento se tenho apenas o
terno do corpo, sem nenhuma troca? Perdi todos os
meus pertences e preciso trabalhar, bastante e devagar, para poder comprar tudo novamente''. Continuei: ``Consegui melhorar, recuperar minha saúde e a esperança de ter mais força, mas ainda estou muito magro e
pálido. Como terei coragem de me apresentar diante da moça e de sua
família?''}\looseness=-1

\textls[-7]{Terminei minha fala com estas palavras, e não ouvi mais comentários sobre o assunto. 
Continuei a visitar a casa deles todas as
noites. Nesse meio tempo, me apresentavam às visitas que apareciam e assim
tive chance de conhecer mais pessoas com quem se relacionavam.}\looseness=-1

Conversavam comigo, e eu contava tudo o que tinha passado por conta da
vontade de ficar junto com os meus irmãos na América. Mas que não 
conseguia realizar o meu sonho e chegar até lá, tinha até 
pegado malária no Amazonas.

\textls[10]{Descobri depois que as pessoas que vieram conversar comigo eram
amigos da família da moça, principalmente do irmão dela. Queriam saber
como eu era e se valia tanto empenho. Eu nada sabia e nem
desconfiava, até que um dia a senhora Kutner adoeceu e o médico a proibiu
de receber visitas. Fui também impedido de entrar na casa.}

\textls[10]{Todas as tardes ia até a porta da casa dos Kutner perguntar se ela
estava melhor, e depois seguia para a pensão. Quando melhorou, perguntou
porque eu não tinha mais aparecido --- e seu marido lhe contou que todos os
dias eu ia até lá, para saber de sua saúde. Ela desconhecia a 
gravidade de sua enfermidade. Pediu então ao marido que me permitisse visitá-la: era 
uma pessoa muito solitária.}

\textls[10]{No dia seguinte, quando cheguei, me disseram que 
ela queria me ver. A senhora Kutner então me perguntou sobre
minha própria saúde, eu disse que estava curado e me sentia muito bem.}

\textls[-10]{Quando se curou, ela voltou a falar sobre casamento,
preocupada com meu futuro. Eu respondi: ``Qual moça irá se
interessar? Não possuo nada e o meu trabalho não é adequado, imagine,
alguém que anda com uma caixa da sorte pela rua, tirando bilhetes como
um \textit{katrinstcik}''.}\looseness=-1\footnote{Em iídiche, \textit{homem do realejo}.}

\textls[10]{A senhora Kutner disse: ``Não tem importância, você é uma pessoa
limpa, honesta e de respeito, não tem nada de patife ou vagabundo, e eu o
considero como um filho mais velho''.}

\textls[10]{Eu me senti tão querido. E continuei a ir à casa deles todas as noites.
Um dia, encontrei lá um jovem sentado à mesa, e conversamos bastante. A
senhora Kutner olhava e sorria, mas eu não compreendia seu contentamento.}

\textls[15]{Na noite seguinte, estávamos sentados à mesa tomando chá, junto estava
sentada uma sobrinha deles. Ela perguntou: ``Tia, você já disse para o
Chaim?'', e a tia respondeu: ``Temos tempo, que pressa é essa?
Antes vamos tomar o chá com pão de ló. Depois, terei força e coragem para
falar com ele''.}

\textls[-5]{Entendi aos poucos que o assunto dizia respeito a mim. O jovem com quem conversei
era o irmão da moça que ela me apresentaria, e a visita à casa de sua família
estava combinada para terça-feira ao anoitecer. O senhor e senhora Kutner iriam junto
comigo, e assim aconteceu.}\looseness=-1

\textls[10]{Na terça-feira, enquanto nos preparávamos para sair, apareceu uma visita para a
senhora Kutner. Ela percebeu então que não poderia nos acompanhar, mas
que Shimen viria comigo. Eu brinquei: ``Shimen, você
parece um casamenteiro, só falta colocar no bolso do paletó um lenço
vermelho''. Saímos os dois para fazer a visita.}

\textls[10]{Fomos muito bem recebidos. Fui apresentado à jovem num ambiente
agradável, e conversamos algumas horas. Todos participaram da conversa: a
moça, a cunhada, o irmão, eu e Shimen. Na hora da despedida, o irmão
dela me convidou a visitá-los mais vezes. Eu agradeci o convite, e na noite 
seguinte fui novamente visitá-los, dessa vez sozinho. Bati
na porta baixinho, com cuidado, e ninguém ouviu --- todos
estavam na sala, e não abriram a porta. Esperei um pouco e em seguida fui embora
para a minha pensão. Tirei o paletó e sentei na cama, meus pensamentos eram tristes. 
Pensei comigo: \textit{Chaim, onde você está se
metendo? Que pouca sorte. Onde já se viu?} De repente, pensei comigo mesmo: 
quem sabe eles não me ouviram bater?}

Coloquei novamente o paletó, o chapéu e voltei para lá. Desta vez, bati mais
forte: escutaram e abriram a porta. Entrei e me sentei na sala junto com a moça, 
conversamos sobre vários assuntos durante algumas horas. Ao final, me
despedi e fui para a pensão dormir, sonhar\ldots

\textls[10]{No dia seguinte, acordei e saí para trabalhar, como todos os dias. No final da tarde,
caminhei até a casa de meus amigos Kutner. A senhora me perguntou se
tinha ido visitar a moça na noite anterior: contei o que tinha
acontecido e ela riu muito, e me perguntou se tinha gostado dela. Eu respondi 
que sim, que tinha gostado, mas que era ainda muito
cedo e precisávamos nos conhecer melhor.}

\textls[10]{Todas as noites, durante alguns meses, fui à casa dela e conversávamos bastante. Contei toda a minha história: do quanto viajei, e
sobre meu sonho de me reunir com meus irmãos na América. Depois de muitas visitas, 
confessei que gostava muito dela, queria assumir o namoro e
marcar o noivado.}

\textls[10]{Ela me respondeu que não devíamos ter pressa. Quem sabe eu já era
comprometido, poderia ter esposa e filhos em algum lugar depois de tantas viagens. 
``Vamos nos conhecer melhor'', lhe disse. ``Posso passar alguns endereços. De meus parentes na Europa, da
cidade onde nasci e cresci, e também dos amigos de Buenos Aires''.}

O irmão dela escreveu para os endereços que dei, e reuniu informações
sobre mim. Os dias passavam enquanto continuava com meu
trabalho e com o namoro, no aguardo das respostas para as cartas enviadas. Não
demorou muito e chegaram retornos positivos: ficamos noivos no dia 14 de
setembro de 1929.

Meu futuro cunhado contou depois que as cartas falavam muito bem de mim. Uma delas dizia: ``Se vocês perguntam sobre o rapaz que conviveu
conosco, podem ficar sossegados e seguir com o casamento. Ele é uma pessoa
muito boa, honesta e educada''.

Casamos no dia 1º de janeiro de 1930, e começamos uma vida nova.

\chapter{Ganhando a vida nos anos 30}
% \addcontentsline{toc}{chapter}{Ganhando a vida nos anos 1930}
% \markboth{Ganhando a vida\ldots}{}

\textls[10]{O irmão da minha namorada trabalhava \textit{com clientela}. Quer dizer que vendia a
prazo, em domicílio. Ele me levou para conhecer, pois gostaria que eu aprendesse o ofício
para também trabalhar com isto. Ele me levou a algumas firmas, garantiu o meu crédito, comprou as
mercadorias. Comecei então a trabalhar por conta própria. Fiz bons
negócios, consegui muitos clientes, aluguei uma sala. }

\textls[7]{Entregava para minha esposa o dinheiro que recebia pelas vendas, que
 armazenava e reservava até o dia do pagamento das mercadorias. Mas de repente estourou a
 Revolução de Outubro de 1930: tive uma reviravolta nos meus negócios
 e perdi muito dinheiro. Vendi novamente para clientes que tinham acabado de
 pagar suas compras anteriores, mas não me pagaram novamente. Fiquei então sem
 dinheiro para pagar as dívidas com fornecedores, e isso me desanimou de trabalhar vendendo a prazo.
 Continuei no ramo até o ano de 1932, quando resolvi mudar para algo mais estável. Durante essa época, morávamos na casa de uma tia da minha esposa.} 

\textls[10]{Um rapaz conhecido, que pretendia abrir uma loja de artigos gerais, estava para alugar um
 local. Seu cunhado iria ajudá-lo, mas ao final se arrependeu e não cumpriu com
 o prometido. O plano do rapaz ficou, portanto, inviabilizado. Enquanto isso, a tia de 
 minha esposa sabia que eu procurava algo desse gênero, e prometeu falar comigo sobre o assunto. 
 Ela me apresentou então ao rapaz, com quem fui conversar. Acabei por ficar com a loja.} 

\textls[10]{Ele me mostrou onde poderia comprar as mercadorias, e comecei a trabalhar. 
Mas a localização da loja não era
boa para comércio, e eu não conseguia vender. Meu lucro não cobria as despesas, 
então saí à procura de outro lugar mais movimentado.}

\textls[10]{Encontrei uma loja fechada. Perguntei à vizinha do lado quem era o dono, e ela me deu o 
 endereço. Acabei alugando o imóvel, que além de loja contava também com uma casa. Ele aceitou minha 
 proposta de acordo, e ainda reduziu o preço. Foi ótimo, pois não precisaria pagar o aluguel de dois locais.} 

\textls[10]{Mudei com a minha família, já tínhamos uma filha de dois anos. Abri
duas portas de frente para a rua e recomecei minha atividade de
comerciante, estabelecido. Eu e minha esposa vendíamos bastante,
o novo lugar era excelente e o dinheiro era suficiente
para cobrir as despesas e cumprir os compromissos.}

Embora o negócio fosse bom, não sobrava dinheiro para comprar mais
mercadorias e ampliar a loja. Os fregueses pediam por outros artigos como 
cimento, tijolos, cal e tintas, tudo referente à construção e
pintura. Mas aos poucos passei a disponibilizar as mercadorias pedidas, e
comprei também um cavalo e uma carroça para entregar as compras.

\textls[10]{Eu comprava as mercadorias e entregava a domicílio. Minha esposa,
enquanto isso, ficava na loja vendendo. Ainda não podíamos ter um
vendedor assalariado. Mais pra frente, ao sentir que já era possível, 
contratamos uma pessoa para nos ajudar. Ele começou a fazer
as entregas, e eu só ajudava quando necessário.}

Não perdi minha vontade de viajar ao encontro de meus irmãos
na América. Quem ler esse relato deve ficar imaginando por que meus irmãos não
poderiam me ajudar, enquanto eu sofria e passava por tudo aquilo. 
Mas eles mal me conheciam, eu só tinha oito anos quando partiram para a América. 
Temia que pensassem que eu era jovem demais, que não sabia ganhar o próprio sustento. 
Estava na América do Sul e talvez não tivesse vontade de
trabalhar: poderia ser um preguiçoso ou um vagabundo. Mas eu não era nem uma coisa
nem a outra, simplesmente não tinha sorte. Tudo que eu tentava fazer
nunca dava certo --- e não queria que eles soubessem disso.

\textls[10]{Quando conheci minha esposa e me casei, ela me ajudou a trabalhar e
acabou minha má sorte. Juntos superamos e conseguimos, com nosso
esforço e trabalho, chegar a melhores resultados, subir
na vida e atingir a independência.}

\textls[-15]{Escrevi então aos meus irmãos que estávamos muito bem com o nosso
trabalho, graças ao bom Deus. Durante os tempos difíceis, as
pessoas falam mal e inventam coisas. Mas sabíamos que
nada daquilo era verdade, somente que é impossível calar bocas maldosas.
Ficávamos então quietos, e Deus nos ajudou. As bocas más emudeceram e mudaram
de opinião, e passaram a dizer que eu era um bom comerciante, esforçado e trabalhador,
e vinham até me pedir conselhos sobre negócios e oportunidades.}\looseness=-1

\textls[10]{É por esse motivo que meus irmãos não me ajudaram quando
eu mais precisei deles. Não por sua culpa, mas porque eu queria
provar que era capaz de me fazer sozinho na vida.}

\textls[-20]{Eu e minha esposa, cujo nome é Marta (Machlia, nos documentos), continuamos
com o nosso trabalho pesado. Economizávamos até em ingressos para o cinema,
pois teríamos que gastar também com as passagens do bonde.
Achávamos melhor gastar em um quilo de carne para o almoço do dia
seguinte, para nós e nossa filha, Bethi, e economizar algum dinheiro.}\looseness=-1

\textls[-15]{Durante essa época, uma cooperativa judaica começou a funcionar. No início, 
concedia empréstimos de duzentos mil réis para pagar vinte mil réis a
cada semana. Para conseguir um empréstimo, era necessário um fiador que garantisse que, 
se eu não conseguisse pagar a tempo, ele pagaria.}\looseness=-1

Pedi a um amigo para assinar, mas ele recusou. Então pedi a outro, e
este concordou. Fiz o empréstimo e investi em carvão, que vendia na loja.
Pensei até em vender e entregar para os judeus a domicílio, mas minha
esposa não gostou da ideia. Disse que eu ficaria conhecido por
\textit{judeu do carvão} e me chamariam assim para sempre, então desisti.

\textls[10]{Graças a Deus, consegui pagar os compromissos assumidos. Durante essa época, um 
de meus fregueses \textit{não judeu}, que comprava madeira e outras
mercadorias destinadas a consertos em sua casa, me fez uma oferta: ``Chaim,
compre a minha loja de artigos alimentícios, uma mercearia, fica bem
perto daqui, na outra rua''.}

Fiquei parado, sem ação pensando \textit{como posso comprar a loja dele se
não tenho dinheiro?} Disse isso a ele, que me respondeu: ``Venderei
para você sem dinheiro. Pode me pagar com mercadorias, eu estou
sempre comprando aqui, vamos descontando aos poucos''.

\textls[-11]{Respondi que pensaria no assunto. Entrei e contei à minha esposa. Conversamos sobre a possibilidade de
fechar o negócio, e resolvemos adquirir a mercearia. Estudamos um
jeito de administrar as duas lojas, e como iríamos trabalhar. Chegamos à
conclusão de que ela continuaria na loja e eu na mercearia, mas
eu contrataria um funcionário. Assim, eu poderia gerir e fazer compras para ambas as lojas, 
assim, eu poderia ajudar a minha esposa quando ela precisasse. E foi o que fizemos. Compramos a mercearia
e o trabalho ficou ainda mais difícil e duro, mas ficamos ainda assim 
muito satisfeitos e achamos que valia a pena pensar no futuro.}\looseness=-1

\textls[5]{Trabalhamos dessa forma durante seis meses, até aparecer um senhor perguntando se 
gostaria de vender a loja. Respondi que poderíamos conversar, era uma possibilidade.
Cheguei à conclusão de que vender a mercearia era conveniente, e teria um bom lucro. 
Então aceitei, fizemos o balanço e ele me pagou. Com o lucro, investi
na minha loja para aumentar o espaço, construí um telhado na parte de trás, aberta, 
e passei as mercadorias que ficavam na frente para lá. Fiz também 
algumas prateleiras, guardei tudo o que havia trazido da mercearia e
recomecei o trabalho junto com minha esposa.}

\textls[-10]{Empreguei um rapaz para ajudar no trabalho, e quando terminei de organizar tudo, fui
até o centro da cidade. Visitei todas as famílias judias e me ofereci para entregar as compras a
domicílio. Comecei a vender-lhes tudo que uma dona de casa precisava:
açúcar, café, azeite, batatas, cebolas, sabão e outros artigos. Levava
comigo uma lista de tudo o que havia na loja, e outra com os nomes e endereços
dos clientes. Anotava os produtos que as senhoras encomendavam e suas 
quantidades. Quando chegava em casa à noite, minha
esposa me ajudava a separar os produtos conforme os pedidos. De manhã
cedo, colocava as caixas de mantimentos na carroça. Meu empregado
atrelava o cavalo e levava tudo direto aos fregueses.}\looseness=-1

\textls[-10]{O esquema deu certo. Um dia buscava as encomendas, no outro meu
funcionário entregava. No dia seguinte cobrava o dinheiro, e já
recebia novos pedidos. Trabalhávamos dia e noite. Ficávamos muito
cansados mas satisfeitos, sabíamos que com o tempo teríamos ótimos
resultados e iríamos longe. Assim trabalhamos até o ano de 1938. Conseguimos 
comprar a casa e a loja, e ficamos livres do aluguel no fim
daquele ano.}\looseness=-1

\textls[-10]{Conheci então um homem que trabalhava em uma
loja de automóveis usados. Disse a ele que me interessava por
esse tipo de negócio, e ofereci 25\% de participação nos lucros além do
investimento na montagem da loja. Seríamos sócios. Ele aceitou, e
começamos a trabalhar juntos.}\looseness=-1

\chapter{Um ativista comunitário}

\textls[10]{Combinei com a minha esposa que ela permaneceria na loja antiga e
continuaria trabalhando com os empregados, e eu ficaria na nova. E assim
foi. Aluguei um local e começamos a comprar carros usados: examinávamos
os automóveis e também caminhões. Se estavam em bom estado, levávamos
para a loja.}

\textls[-20]{Aconteceu que o meu sócio não era honesto comigo. Como era ele o
entendido, tornou-se o comprador e também o vendedor. Então 
combinava com o proprietário do veículo para que ele pedisse mais
dinheiro, e ficava com a diferença que eu pagava a mais. Eu ainda não
tinha experiência nesse ramo. Quando ele dizia o que eu devia
pagar, eu acreditava e pagava.}\looseness=-1

\textls[-20]{Trabalhei com ele desse jeito por seis meses,
até aprender melhor. Então observei e constatei que estava sendo roubado, 
e que não concordava mais com os preços que ele acertava. Comecei eu 
mesmo a comprar. Depois de um ano dissolvi a
sociedade, paguei a parte dele e contratei outra pessoa. Pagava todo mês
um ordenado, e me tornei o único dono do negócio.}\looseness=-1

\textls[-20]{Resolvi por fim liquidar a loja, e aluguei um lugar na cidade. Mudei com a
minha família: já tínhamos três filhos, um menino e duas meninas. Começava o ano de 1942, e passei a trabalhar sozinho, minha esposa passou a 
cuidar de nossa casa e dos nossos filhos. A vida mudou para
melhor, e minha esposa tornou-se uma dona de casa igual às outras senhoras
judias.}\looseness=-1

Meus negócios foram melhorando cada vez mais: comprei então uma casa na
cidade e, pouco depois, um palacete para onde mudamos com a família. Eu já
possuía um belo automóvel, íamos ao cinema, ao teatro, podíamos tirar
férias e viajar para a praia com nossas crianças. Durante alguns
meses, tomávamos banho de mar.

Tornei-me uma pessoa importante, proprietário, e podia mandar os meus
filhos para estudarem em uma escola judaica. Nesta época, estavam
escolhendo um conselho administrativo para a escola judaica e me
convidaram para participar. Aceitei e recebi a indicação para ser o
tesoureiro. Fui eleito para este cargo. 

\textls[10]{Trabalhei para a escola durante
quatro anos. O presidente se chamava Idel Fainzilber, e a secretária,
Berta Margolis. Juntos no novo conselho, melhoramos muita coisa:
conseguimos pagar em dia o salário dos professores e aumentamos o número
de alunos para oitenta. Contratamos também mais dois professores brasileiros e uma
professora de iídiche, seu nome era Sara Mancovetsky. Durante estes
anos, os professores de iídiche que lecionaram na escola se chamavam
Burstein, Alpern, Oksman e Bekin.}

A escola funcionava num prédio comprado pela comunidade judaica, formada por 
judeus sionistas e da esquerda progressista, chamado
Círculo Israelita. O conselho do Círculo incluía na composição ambas
as partes para participar das decisões. Houve então uma ruptura, e os
sionistas em sua maioria saíram do Círculo Israelita: adquiriram outra
sede e formaram em separado uma nova associação. Mas eles não alcançaram
o esperado sucesso, porque a sede ficava longe do bairro onde a maioria
dos judeus moravam, e naquela época poucas pessoas possuíam carro --- além de já
estarem acostumados ao Círculo Israelita. Lá havia uma boa
biblioteca e o espaço externo servia para a prática de
esportes. Aos poucos, esse grupo voltou para o Círculo.

\textls[10]{O prédio acabou por ser alugado para um quartel do governo do estado. Assim, passaram
alguns anos até o local ficar desocupado. O grupo dissidente veio para
uma reunião do conselho escolar, e sugeriu que a escola 
ocupasse o prédio vazio. Bastava adaptar o que fosse necessário
para seu bom funcionamento. Muitos membros do conselho escolar
estavam de acordo e achavam que devíamos aceitar. Eu ouvi
tudo com muita atenção até pedir a palavra. E disse também estar de acordo, mas com uma condição: que a escritura do prédio ficasse em nome da escola, pois pertencia à 
coletividade judaica da cidade. Disse ainda: ``O conselho escolar muda todo ano. Os membros no momento são favoráveis à ideia e vocês aprovam, mas caso daqui a alguns anos o conselho eleito não agrade a vocês, e queiram pedir a casa de volta? A escola então não terá para onde ir, e o espaço que ela hoje ocupa estará sendo usado para outras atividades, e seria diretamente 
prejudicada. Sugiro então que seja providenciada a escritura no tabelião,
de modo garantir o futuro da escola''.}

\textls[10]{Eles não aceitaram o meu ponto de vista e durante um mês não
compareceram às reuniões. Não houve acordo, passou
um tempo e fiquei pensando em alguma maneira de nos entendermos.
Surgiu uma ideia, e chamamos o grupo para um novo encontro. Quando nos
vimos, disse: ``Vamos passar a escritura da casa
para a escola, mas acrescentaremos um ponto: enquanto existir escola
judaica em Recife, a casa pertencerá à escola e, se por algum motivo
ela parar de funcionar, o prédio voltará a pertencer novamente ao
grupo''. Eles aceitaram a proposta e concordaram. Passamos a escritura
em nome da Escola Israelita, e assinamos como representantes da escola:}

\begin{itemize}
\item \textit{Presidente} Idel Fainzilber
\item \textit{Tesoureiro} Chaim Novodvorsky
\item \textit{Secretária} Berta Margolis
\end{itemize}

Consertamos o prédio e mudamos a escola. Adaptamos tudo em função dela,
e eu continuei trabalhando lá até 1946.

Enquanto resolvíamos isso, a Europa estava em guerra. Era a Segunda Guerra Mundial, 
e em 1943 vários navios brasileiros foram
afundados perto do Recife, por alemães ou fascistas italianos. O governo
brasileiro apelou para o povo ajudar a defender o país. Para isso,
o Brasil precisava aumentar o número de aviões e treinar os pilotos.

Já naquela época, viviam no Brasil pessoas de várias nacionalidades. E,
atendendo ao governo brasileiro, todos começaram a angariar dinheiro para
comprar os aviões. Nós, os judeus do Recife, também queríamos
contribuir, junto com os judeus do Rio de Janeiro e de São Paulo.
Enviamos cartas para as comunidades pedindo orientação de como
deveríamos proceder, mas a resposta demorou muito: então resolvemos
sozinhos doar um avião para treinar os pilotos. Formamos uma comissão, e
eu fui escolhido para me encarregar do caso.

No dia que o avião chegou no Recife, fomos todos ao aeroporto e
convidamos o governador de Pernambuco, na época Agamenon Magalhães, além do
prefeito Novais Filho e de outros políticos. O prefeito do
Recife começou seu discurso com estas palavras: ``Nossos amigos judeus
da nossa cidade do Recife doaram um avião para o governo, para
ensinar e treinar mais pilotos brasileiros a fim de podermos nos defender
de nossos inimigos nazistas e fascistas, que querem mudar o
mundo. Estou lhes agradecendo em nome de nosso governo. Muito obrigado.''\looseness=-1

\textls[-10]{Outras personalidades também falaram. Minha filha Bethi, que tinha
então 12 anos, foi a primeira a subir no avião para voar. Depois também fui eu, pela primeira vez na vida. O nome escolhido para ser escrito no avião foi pensado em honra ao grande brasileiro Joaquim Nabuco.}\looseness=-1

\textls[10]{Nós, judeus, somos bons amigos do povo brasileiro. E eles também são bons
amigos dos judeus que vivem aqui. No Brasil, os judeus são muito
considerados. Trabalharam e fizeram muitos investimentos, conseguiram ficar bem de vida materialmente. Quando os navios foram afundados, o povo brasileiro ficou muito revoltado contra os alemães e
os italianos que viviam aqui, inclusive quebraram suas lojas. Mas nós, judeus,
tivemos nossos estabelecimentos respeitados, sabendo que somos amigos e
doamos um avião.}

\textls[10]{Os manifestantes que saíram às ruas para quebrar as lojas paravam em
frente às nossas e proferiam discursos. Afirmavam que a
coletividade judaica é boa amiga do povo brasileiro.}

Depois, chegaram ao Recife soldados americanos. Entre eles, muitos
militares eram judeus. Eu entrei na comissão para lhes dar apoio moral,
material e também religioso. Organizamos, no Círculo Israelita,
recepções e festas para eles. As famílias os convidavam para
participarem das festividades tradicionais de Pessach\footnote{Festividade
  comemorada durante oito dias, no início do mês de Nissan, que
  corresponde à primavera no hemisfério norte. Relembra e celebra a
  saída dos judeus do Egito. {[}\textsc{n.\,t.}{]}} e outros feriados judaicos. Nossa comissão
pediu aos soldados judeus americanos que fornecessem uma lista com o
nome de todos (e eram muitos), para que as famílias
judias avaliassem a administração dos convites para o primeiro
\textit{seder}\footnote{Dá-se o nome de \textit{seder} à cerimônia de celebração de
  Pessach, que ocorre durante o jantar com comidas que aludem à festividades. A saída do Egito é celebrada por meio de narrativas da história, canções e jogos. {[}\textsc{n.\,t.}{]}} de Pessach. As famílias os receberam em suas casas para fazê-los sentirem-se num ambiente familiar, embora longe de casa.

\textls[-10]{Fui o último a permanecer na sede do Círculo Israelita, e ainda ia passar
 para buscar meus dois convidados. Quando
 estava de saída, apareceram cinco marinheiros de um navio que tinha
 acabado de atracar no Recife ao anoitecer: chegaram no Círculo quando
 já não havia mais ninguém. Então, resolvi levá-los todos para minha casa.
 Fui andando com os sete marinheiros e uma vizinha nos viu de longe e
 disse para minha esposa: ``Olha, Marta, quantas pessoas o seu Chaim está
 trazendo para casa no \textit{seder}''. Minha esposa respondeu: ``Não se
 preocupe, o que eu preparei será suficiente para todos, cozinhei
 bastante e não faltará comida''.}\looseness=-1

Comemoramos com muita alegria o primeiro
\textit{seder} de Pessach junto aos nossos convidados. Eles nos agradeceram muito e
elogiaram a atenção recebida. Nos trouxeram algumas
\textit{Hagadot} de Pessach\footnote{Livros utilizados nas celebrações de
  Pessach. Contêm a narração da história do êxodo do Egito, bem como
  as canções e orações dessa festividade. {[}\textsc{n.\,t.}{]}} distribuídas pelo Exército
Americano, com as quais continuamos a rezar na minha casa durante muitos anos.

\textls[10]{Nos dias de \textit{chol hamoed}}\footnote{Festividades mais longas são divididas
  entre dias considerados \textit{chag}, em que se aplicam as mesmas exigências
  de dias como o \textit{shabat}, e dias intermediários, denominados \textit{chol hamoed},
  em que apenas se mantém os costumes da festa.} \textls[10]{Pessach, chegou no
  Recife o rabino Boim. Organizamos o terceiro \textit{seder} no Círculo
  Israelita e quase toda a coletividade compareceu: também os soldados e
  marinheiros americanos resolveram fazer uma apresentação para as
  crianças da Escola Israelita, em seu acampamento que ficava no aeroporto
  chamado de Campo do Ibura. Foi uma bela festa naquele dia.} 

\textls[-14]{Os militares judeus mandaram vir os grandes carros do Exército Americano. Subimos nos veículos militares e viajamos até o acampamento: as crianças e adultos, alunos e professores de iídiche, professores brasileiros, eu como tesoureiro, o presidente e a secretária, os pais dos alunos.
Foi muito emocionante, participamos e cantamos juntos, tomamos coca-cola pela primeira vez na
vida. Fiquei todo o tempo junto com os soldados.}\looseness=-1

O contato com eles foi muito bom --- até que a Guerra terminou, graças a
Deus, e eles voltaram para América. Continuei recebendo cartas, pena
que não as guardei. Lembraria o nome de todos: não pensei que
um dia escreveria a minha biografia.

\chapter*{Com meus irmãos, na América}
% \addcontentsline{toc}{chapter}{Com meus irmãos, na América}
% \markboth{com meus irmãos\ldots}{}

\textls[-20]{No ano de 1946, finalmente fui com minha esposa para América, visitar
meus irmãos e irmãs.}\looseness=-1\footnote{Chaim foi o filho mais novo de Eli e Chaye Tzirl Novodvorsky. Teve sete irmãos: Isaac, Morris, Sam, Louis, Gershon e Jim (Shimen), além de duas irmãs, Sheindel e Chandel.} \textls[-20]{Escrevi uma carta para o meu irmão Jim Novy. Este
era o seu nome na América. Antes, na Europa, seu nome era Shimen
Novodvorsky. Ele me respondeu dizendo que viajasse no mês de junho,
pois minha sobrinha, filha de uma irmã, se casaria e encontraríamos
toda a família. Minha irmã morava no México, mas o casamento
se realizaria no Texas, na casa do meu irmão Jim. O noivo era um rapaz
americano judeu.}\looseness=-1

\textls[10]{Começamos os preparativos para a viagem. Nossos filhos ficariam com
os tios, por parte da minha esposa. No dia 9 de junho de 1946, tomamos um avião pequeno até o Pará, e ali entramos em um outro grande chamado \textit{Constellation}. Fizemos escala em Miami, e depois num terceiro avião até Dallas, no Texas.}

\textls[10]{Em Dallas, tomamos mais um avião pequeno até o local onde seria
realizado o casamento, e onde meus irmãos moravam. Durante a viagem,
enfrentamos um forte temporal, e quase não chegamos a tempo de encontrar a
família.}

\textls[10]{O avião não conseguia pousar por conta do vento em Austin, onde
 deveríamos ficar, então tivemos que voltar. Descemos e aguardamos duas
 horas no aeroporto. Indagamos se não existiria a possibilidade
 de viajar de trem, não queríamos mais ir de avião.} 

\textls[7]{Como sempre existe gente que interfere em qualquer situação, aqui também
apareceu um senhor judeu e disse: ``Ouvi que vocês estão querendo ir de
trem'', e explicou que a Companhia de Aviação não gostava que os
passageiros abandonassem o avião porque isso não era uma boa propaganda
para eles.}

Em seguida, fomos chamados para entrar no avião, e disseram que já
partiríamos. Mas era mentira, e ficamos esperando ainda uma hora até que o
tempo melhorou e voamos até Austin, no Texas.

Quando chegamos, um senhor nos mostrou algumas pessoas
e avisou que estavam esperando por nós. Não conhecia os meus irmãos, mas
quando nos aproximamos deles nos abraçamos e choramos muito emocionados. 
Até esqueci de apresentar a minha esposa. Na confusão, ela mesma se
apresentou a todos. Minha família que me aguardava estava em 
mais ou menos trinta pessoas entre irmãos, irmãs, cunhados, cunhadas e
outros parentes. Eu não os conhecia, pois na época que eles saíram da
Europa e viajaram para a América eu era um menino de apenas oito anos.
Tinham se passado 34 anos desde então.\looseness=-1

\textls[-10]{Entramos nos carros e fomos levados até a casa de meu irmão Jim,
onde ficamos hospedados até o casamento. Ao chegar, fui com ele conhecer
sua empresa. Em seu escritório, lhe disse que
precisava ir até o banco, pois já estávamos ali há dois dias e eu achava
que o dinheiro que tinha transferido do Brasil deveria estar na conta.
Admirado, ele perguntou: você mandou dinheiro para cá? Respondi que sim,
tinha mandado 20 mil dólares. Queria comprar mercadorias para minha
loja de automóveis, peças para Ford, Chevrolet, caminhões 
e os acessórios necessários.}\looseness=-1

\textls[-20]{Rumamos ambos até o banco. Assim que chegamos, informaram ao Jim que tinha
chegado um dinheiro para alguém Novodvorsky, mas em seu nome. ``Sim'',
disse ele, ``é para Chaim, meu irmão''. ``E você tem mais um
irmão?'', perguntaram. ``Mas o nome dele é diferente do seu, você se
chama \textit{Novy} e ele se chama \textit{Novodvorsky}''. Jim falou então: ``A diferença é
que ele conservou o sobrenome da família, como todos nos chamavam na
Europa''.}\looseness=-1

Todos os dias eu acompanhava meu irmão até seu escritório. Como ele
se mantinha ocupado, eu saía sozinho para a rua e ficava
olhando as lojas. Acabei por encontrar uma que vendia exatamente as peças e
acessórios que eu precisava para a minha. Entrei e perguntei
os preços mas, como não falava inglês, pedi uma lista telefônica a uma
pessoa que estava lá. Achei o nome e o número de telefone do meu irmão,
mostrei para o vendedor com a mão e pedi para que ele ligasse para lá.
Ao tocar, ele me passou o fone para que eu falasse. Quem
atendeu foi a secretária dele, que me perguntou quem queria
falar. Eu disse: ``Chaim'', e ela ``Chaim de onde?'' Respondi: ``do Brasil'', porque
parece que havia ali outro Chaim. ``Espere um momento'', respondeu. Fiquei
parado, sem entender o que ela tinha falado, mas
continuei segurando o fone.

\textls[10]{Aguardei um pouco, até meu irmão falar comigo. Perguntou
onde eu estava, no que respondi brincando que estava em uma loja e queriam me
prender em uma cela. Ele achou engraçado, e perguntou admirado de onde eu
conhecia essa expressão. Disse: ``Eu sei o que isso significa, mas a
pessoas aqui não entendem as minhas perguntas. Estou em uma loja de
acessórios de automóveis, com peças que eu preciso comprar''. 
Meu irmão pediu então para falar com o vendedor: ambos iriam se entender. 
Peguei na mão dele e coloquei o fone, e os
dois iniciaram uma conversa. Ficou claro que se conheciam.}

\textls[-12]{O vendedor ficou contente, e disse para meu irmão que pelo jeito
eu entendia bem sobre as peças e acessórios dos automóveis Chevrolet e
Ford. Jim respondeu que no Brasil eu tinha uma loja no ramo, e queria
comprar estas mercadorias para revender. Pediu então para designar uma pessoa
para me mostrar tudo: ele viria depois para fazer a compra junto.
Disse também ao vendedor que eu era seu irmão menor e residia no Brasil.
Ao desligar, uma pessoa entendida no assunto apareceu e me acompanhou, mostrando todos os
artigos da loja e o que me interessava comprar.}\looseness=-1

\textls[10]{Nessa época, na América, estes produtos estavam em falta. A fabricação era
reduzida, e o governo havia proibido que as peças fossem vendidas para
fora, não podiam ser exportadas.}

\textls[10]{Meu irmão então escreveu para um bom amigo dele, Lyndon Jonhson,
em Washington, para que conseguisse a
permissão da saída de minhas mercadorias. Naquele tempo, era senador. Ele
respondeu com um telegrama que servia como comprovante, e dizia que eu poderia
embarcar e mandar tudo para o Brasil.}

\textls[10]{No momento em que escrevo a minha autobiografia, Lyndon
Jonhson é o presidente dos Estados Unidos da América do Norte --- nesta
data, 14 de setembro de 1964.}

\textls[15]{Chegou o dia do casamento da minha sobrinha. Foi muito bonito. Minha
 esposa e eu conhecemos toda a família, até os conterrâneos da Europa
 meu irmão nos apresentou. Ele depois telefonou para um conterrâneo
 em Chicago, pedindo que reservasse um hotel para nós e que me levasse
 aos lugares onde eu poderia encontrar mercadorias que ainda estavam
 faltando em minhas compras.} 

\textls[10]{Tudo isso se passou logo depois da Segunda Guerra Mundial, e muitas mercadorias estavam em falta. Quando chegamos em Chicago, o amigo de meu irmão já estava nos esperando e nos
levou para o hotel. Depois, saímos juntos para procurar as mercadorias.}

Ficamos em Chicago por três meses, até eu conseguir comprar tudo. Tive que
ligar para meu cunhado no Brasil, e pedir que enviasse mais dinheiro, pois ele ficou tomando conta do meu negócio. Ele mandou o dinheiro: 
paguei então tudo que comprei e despachei para o Recife.

\textls[10]{Voltamos para Austin, e ficamos com os irmãos e suas famílias ainda mais
um mês. Comprei do meu irmão um grande lote de mercadorias, e tive
sorte pois foi justo na semana que os preços baixaram. Meu irmão Jim
fornecia peças para o Exército Americano, e me vendeu o que sobrou
antes de devolver as peças. Escolhi o que servia para minha loja e
despachei também.}

\textls[10]{Enquanto estivemos com meus irmãos e irmãs, a conversa sempre
girou em torno do mesmo assunto: nos convencer a mudarmos para a
América. Eles poderiam providenciar os documentos necessários para toda a 
minha família, e obter os vistos para nossa entrada legal. Meu cunhado
David me dizia: ``Chaim, nós vamos abrir para você uma loja, e eu ficarei
junto alguns meses até você aprender a vender e falar inglês''. Então
meu irmão Jim respondeu: ``Como você quer ensinar, não está vendo que
ele é capaz de ensinar a todos nós como se compra e vende? Você viu que
ele sem falar inglês geriu grandes negócios por aqui, e ainda
exportou para o Brasil? Você ainda está querendo ensinar algo?
Deixe somente que ele concorde e aceite vir para cá, já será ótimo
ficarmos todos juntos''.}

Eles queriam que eu prometesse vir me reunir
com a família. Mas eu disse: ``Não posso prometer nada. Quando 
voltarmos para o Brasil, vamos pensar e resolver sobre isso''.

\chapter{Uma vida no Brasil}

\textls[-20]{Ao voltar para o Brasil, encontrei minha nova loja pronta. Meu
cunhado tinha providenciado tudo, em todos os detalhes. Teríamos
o espaço necessário para guardar as mercadorias que comprei na América. 
E em algumas semanas, elas começaram a chegar. Contratei
uma pessoa para expor, colocar nas
prateleiras e elaborar um catálogo de preços a partir de cada item.}\looseness=-1

\textls[5]{Naquele tempo, no Brasil, as mercadorias que trouxe estavam em falta, 
então ganhei um bom dinheiro. Mas eu e minha
esposa resolvemos não mais pensar em mudar para a América, pois 
aqui vivíamos bem, bem sucedidos material e socialmente. Com o tempo, 
teríamos bastante dinheiro para visitar sempre meus irmãos e irmãs: além do mais, 
seria bem melhor ir para lá passear. Depois de tantos anos de privações, não
tínhamos mais necessidade de começar novamente num outro país, com 
novo idioma e novos negócios. E, tendo feito contato com as empresas na América, 
era só escrever pedindo as mercadorias das quais precisava que me
enviariam. Consegui um bom crédito nos bancos, e passamos a desfrutar de
uma vida confortável no Brasil. Uma vida feliz, junto com minha
esposa e nossos filhos.}

Em 1949, nossa filha mais velha Bethi ficou noiva. E, em 1950, ela se
casou. Nosso genro Rubens tinha se formado como dentista, mas um
profissional no início de carreira demora para se tornar conhecido. Ele não ganhava o suficiente.

Em 1951, viajei novamente para América. Surgiu uma
nova lei que exigia licença para importação de mercadorias: e eu já tinha excedido a minha cota. Fui então pessoalmente, despachei tudo que eu precisava de lá, e passei um mês com meus irmãos e minha família.

Quando voltei ao Brasil, soube que o meu genro se correspondia 
com um amigo que morava em São Paulo. Eles tinham estudado
juntos, se formaram no Recife. Esse amigo viajou e escrevia nas cartas
que São Paulo era muito bom para os dentistas, que tinha começado um
consultório e os ganhos eram excelentes. Meu genro ficou
entusiasmado, e resolveu ir até lá para encontrar alguma oportunidade na área. Ele gostou do que viu, e voltou para o Recife decidido a mudar para São Paulo, junto com minha filha e
minha primeira neta Jeanete.

Senti tanta saudade que alguns meses depois fui visitá-los, e
percebi que não era exatamente como o amigo tinha escrito. Esse amigo era solteiro, e para ele era realmente suficiente o que ganhava. Mas meu genro precisava ganhar mais, tinha uma família para sustentar.

Minha filha foi trabalhar em um escritório para completar a renda, e 
meu genro não fazia bons negócios. Implorei: ``Filhos, voltem para
Recife!'' A casa deles estava desocupada, e eu não tive vontade de
alugá-la enquanto a situação não se estabilizasse. Mas não adiantou pedir, eles resolveram enfrentar a vida em São
Paulo com a esperança de melhorar. Minha filha disse:
``Papai, você mesmo nos contava o que passou na vida. E que superou
todas as dificuldades. Nós somos jovens, vamos enfrentar e
vencer também''.

\textls[-10]{Eu não podia discutir e insistir para que voltassem ao Recife, então 
voltei sozinho. Pensei, pensei\ldots{} o que fazer? Resolvi vender tudo o que
possuía e me mudar com toda a minha família para São Paulo. Mandei minha
esposa e meus dois filhos para conhecerem a cidade. Quando 
voltaram para o Recife, disseram que não gostaram de lá. 
Que já estavam acostumados em Recife, onde todos nos conheciam,
tinham amigos. Mas minha filha mais nova, Cecília, que tinha então 12
anos, me disse: ``Papai, se você deseja se mudar para São Paulo, nós
vamos todos juntos''.}\looseness=-1

Comecei então a recordar como a vida fora ingrata para meus pais, que não
tiveram a sorte de conviver com os filhos, de ficarem juntos, verem os
netos nascendo. Cada filho tomou um rumo, e se dispersaram
pelo mundo. Não queria que acontecesse o mesmo comigo. Meu filho e
minha filha ainda eram pequenos, mas quando crescessem também
acabariam tomando outro rumo, enquanto eu e minha esposa acabaríamos ficando sozinhos.
Achei que o melhor a fazer era mudar para São Paulo, onde já
moravam minha filha Bethi, meu genro e minha neta, assim poderíamos voltar à alegria de estarmos juntos.\looseness=-1

\pagebreak

\textls[10]{Fui aos poucos vendendo meus negócios. Livrei-me de tudo e parti para São
Paulo, para buscar outro empreendimento. Minha esposa Marta
permaneceu no Recife com meus filhos Elias e Cecília, que estavam
estudando e não podiam perder o ano letivo.}

\chapter{O cinema no Bom Retiro}

\textls[10]{Chegando em São Paulo, vi num anúncio do jornal que um cinema estava à venda no Bom 
Retiro, bairro onde os judeus moravam em São Paulo. E 
resolvi comprar.}

Escrevi para minha esposa pedindo que viesse, por enquanto sozinha, e
deixasse nossos filhos no Recife com os tios. Quando chegou, compramos 
um apartamento no Bom Retiro e arrumamos toda a casa. Ao final, ela foi 
buscar Elias e Cecília e os trouxe para São Paulo.

Trabalhava com o cinema, mas não dava certo: todo mês tínhamos  
déficit, e fui obrigado a completar com o dinheiro que trouxe do Recife.

\textls[-15]{Um dia encontrei um conhecido que morou no Recife, Isaac Schuster. 
Tinha se mudado para São Paulo há alguns anos, pois o clima do 
Recife não fazia bem para a sua saúde.
Conversamos, contei sobre meus problemas e o que aconteceu com meus negócios. Isaac me aconselhou a abrir uma loja
de móveis, ele me mostraria como organizar e fazer dar
certo. Ele tinha uma loja numa cidade perto de São Paulo, em São
Caetano, mas para chegar lá era preciso tomar um trem. Ele estava em vias de terminar a construção de uma casa, e poderia alugá-la pra mim junto com uma loja.}\looseness=-1

Enquanto isso, apareceu uma loja de móveis já pronta para comprar. 
Fui com minha esposa ver e gostamos muito. Visitei-a durante alguns
dias, vi que tinha um bom movimento e estavam vendendo bastante, então
combinei com o dono para fazer o balanço e fechar o negócio.

Procurei por meu amigo Isaac Schuster e lhe contei sobre a loja que
iria comprar, porque antes estava negociando com ele. Pedi que
me deixasse livre da palavra que tinha dado. Ele me respondeu que
estava contente por eu comprar uma loja que já tinha freguesia própria e não estava aborrecido comigo, pois poderia alugar para
outra pessoa sem prejuízos. No dia marcado para fazer o
balanço, fui com minha esposa até a loja que iria comprar, pronto
para pagar o que combinamos e assumir o negócio. Mas o dono
tinha se arrependido e não quis mais vendê-la. Era uma pessoa que
não assumia a palavra, e toda a negociação deu em nada. Voltei para o
meu conhecido Isaac e contei o que tinha acontecido. Ele respondeu que não tinha mudado de opinião, e se
eu quisesse ele alugaria a casa e a loja --- e ainda ajudaria a me
estabelecer.\looseness=-1

\textls[15]{``O principal'', disse ele, ``é ver vocês bem instalados''. E assim foi
feito, ele foi comigo comprar os móveis, ajudou a colocar nos lugares
certos e me mostrou como deveria trabalhar.}

\textls[10]{Meu filho ficou administrando o cinema. De manhã ele estudava, e depois
se dedicava ao trabalho. Mas isso até eu conseguir
vendê-lo, pois estava perdendo uma boa quantia de dinheiro. A nova loja estava indo
bem mas, depois de alguns meses, o movimento ficou mais fraco porque eu
só vendia à vista, não queria parcelar.}

\textls[-15]{Como não tinha outro jeito, comecei a vender a prazo. Como não conhecia as pessoas, pedia
informações. Mas mesmo assim eu perdia, porque os fregueses mudavam de
endereço e eu não conseguia mais encontrá-los. Só podia ir aos domingos, estava a semana inteira ocupado na loja. Meu costume aos
domingos era ficar com a família, almoçar com os filhos e a neta, todos
juntos na minha casa. Então acabei perdendo até o prazer de aos domingos desfrutar da companhia de meus familiares,
compartilhar os bons momentos com eles. Resolvi então vender somente à
vista. Embora o movimento fosse menor, ao invés de ir procurar os clientes
em suas casas, ficaria junto com minha família aos domingos.}\looseness=-1

\textls[10]{Isaac, que possuía uma loja de joias e
outros artigos de ouro, me aconselhou que eu também trouxesse essas mercadorias para a 
loja. E, além disso, providenciasse um relojoeiro para
consertos de relógios dos clientes. Isaac
me levou aos lugares onde ele sempre comprava e confiava, para poder escolher
bem sem ser enganado. Comecei então a vender relógios e joias. Depois, completei as prateleiras com rádios e outros aparelhos úteis para uma casa.}

Existem pessoas que ficam com inveja dos outros, e têm olho gordo. Passou
um judeu diante da loja, parou e disse para minha esposa: ``Veja
só! Faz pouco tempo que a loja foi aberta e ele já conseguiu ampliá-la
vendendo relógios e joias?'' Minha esposa sem perder tempo respondeu:
``Não tenha dúvida, ele não investiu o dinheiro que ganhou em São Paulo,
investiu o dinheiro que trouxemos do Recife!'' Os judeus paulistas achavam que as pessoas que chegavam do Nordeste,
como nós, eram uns pobres coitados morrendo de fome, mas lá no Recife os
judeus viviam muito bem, levavam uma vida boa e decente.

\textls[10]{Mas o fato é que eu não me sentia satisfeito com a loja. Em primeiro lugar,
porque estávamos eu e minha esposa ocupados durante a semana toda. Chegávamos às 
sete horas da manhã e fechávamos às sete horas da noite,
às vezes até às oito. Só podíamos ficar juntos com os filhos aos
domingos, e mesmo isso me foi tirado porque eu precisava
fazer as cobranças: os clientes se mudavam de endereço e
não se preocupavam em pagar as prestações.}

\textls[-10]{Comecei de novo a pensar em mudar de loja e de ramo. Ter pelo menos
algum tempo de folga para ficar junto de nossos filhos, que também
estavam ocupados com seus estudos. Encontrei um jovem que também tinha
chegado do Recife, já nos conhecíamos, éramos vizinhos. Ele me contou
que estava revendendo malhas de lã para
homens e mulheres, ele disse que era um bom negócio.
Combinamos de abrir uma fábrica e trabalhar com malhas. Dinheiro ele
não tinha, então eu investiria o que fosse necessário e ele, que já entendia do ramo, faria a parte
técnica e viajaria como vendedor. Eu, que entraria com o dinheiro,
teria 60\% do lucro, e ele 40\%. Assim pensei quando fiz a
sociedade com ele.}\looseness=-1

\textls[5]{Alugamos um salão no Bom Retiro e compramos as máquinas para
fabricação das malhas. Então descobri que meu sócio só
sabia comprar fios de lã: discernir quais cores eram importantes, e vender.
De máquina ele não entendia nada. Tanto que compramos algumas que não serviram
para nós. Tivemos que vendê-las e perdemos um pouco de dinheiro. Fui atrás de pessoas que trabalhavam com esse tipo de malha, que nos ensinaram como iniciar o trabalho.}

\textls[10]{Passaram-se alguns meses até resolvermos tudo. Meu sócio desanimou e
falou: ``Senhor Jaime, não está vendo que não está dando certo?
Talvez seja melhor desistir e terminar com a sociedade. O que tiver de
prejuízo, lhe pagarei aos poucos''.}

\textls[10]{Eu respondi:``Do que você está falando? Não podemos fazer isso. Eu
apliquei o último dinheiro que tinha, precisamos achar o caminho certo
para evitar perder tudo''.}

Conseguimos trabalhar. Entregamos os fios de lã a uma pequena fábrica
que fazia o trabalho para outras fábricas maiores. Com suas máquinas,
teciam os fios de lã e só pagávamos pelo trabalho: eram eles quem fabricavam
o tecido para nós. Colocamos operários para cortar e costurar as malhas,
meu sócio saiu na rua para vender, e justamente naquele ano o inverno em São Paulo foi
de um frio intenso. Fizemos bons negócios, mas meu sócio
começou a ficar um pouco preguiçoso. Estava acostumado a ser vendedor
e não dono, somente saia para vender entre nove e dez horas da manhã, e
quando vendia algumas malhas achava que já era o suficiente. Quando
temos uma fábrica, precisamos pensar diferente e trabalhar para subir na
vida. Enquanto isso acontecia, eu dirigia a fábrica prestando
atenção para dar tudo certo.\looseness=-1

\textls[-15]{Meu sócio ficou doente e teve que ficar de cama alguns meses, e eu
continuei trabalhando sozinho. Ainda bem que meu filho Elias vinha à
tarde para me ajudar, pois de manhã estudava. Um dia meu filho
disse: ``Papai, não sei porque você precisa de um sócio, você está
fazendo tudo sozinho''.}\looseness=-1

Eu passei a entender de toda a rotina da fábrica. Quando meu sócio voltou a
trabalhar, sugeri dissolvermos a sociedade. Eu e
meu sócio não combinávamos: ele não tinha ambição de batalhar como eu,
talvez porque já fosse nascido no Brasil. Trabalhamos juntos
ainda durante um ano, depois conversamos e eu disse: ``Se quiser,
pode ficar com a fábrica, e pagar a minha parte. Pode até colocar outro
sócio no meu lugar''. Ele, então, me respondeu que não queria nada disso
porque não saberia dirigir uma fábrica, só gostava de trabalhar com
vendas, preferia que eu pagasse para ele a parte que lhe pertencia.

\textls[-20]{Acertamos tudo e permanecemos bons amigos. Com o dinheiro recebido, ele abriu uma fábrica de fiação junto com um italiano de quem se tornou sócio, e deu
certo. Eu fiquei contente que tudo se resolveu bem. Ele está bem de
vida e dou graças a Deus pelo bem estar de meus amigos.}\looseness=-1

\chapter{Negócios em família}

Continuei na fábrica e meu filho ajudou a dirigi-la. Era o ano
de 1957, estávamos crescendo e progredindo. Meu filho
terminou os estudos, e começou a se preparar para cursar engenharia na universidade.

\textls[-7]{Na mesma época, foi inaugurada a filial do Banco Nacional do Norte com
matriz no Recife, em 1958. Fui convidado para o
banquete de abertura. Participei do evento e encontrei lá uma pessoa conhecida do Recife, o presidente do Banco
Indústria e Comércio de Minas Gerais, Manoel Teixeira Bueno.
Nos olhamos de longe, até que nos aproximamos e ele me perguntou:
``Chaim, o que você está fazendo aqui?'' Respondi que morava agora em São
Paulo, e tinha uma pequena fábrica.}\looseness=-1

Não sabia que ele era o homem mais importante do Banco Nacional do
Norte. Ele chamou o gerente do banco, Sebastião de Carvalho Mergulhão, e
me apresentou dizendo que eu era um bom amigo e cliente antigo do
Recife, e que podia me atender em tudo que eu precisasse.

\textls[-15]{Depois de um ano e meio em São Paulo, Mergulhão foi chamado de volta
para o Recife. Tinham comprado outro banco em São Paulo, e
ele foi nomeado subdiretor de ambos. Em seu lugar, ficou Luiz Gonzaga da
Silva Tescari, e os dois me ajudaram muito em tudo que precisei.
Financiaram uma máquina que trouxe da América, porque sabiam que eu era
um homem honesto em São Paulo, no Recife e nos outros lugares onde estive.}\looseness=-1

Meu filho Elias acabou se profissionalizando como um dono de fábrica junto comigo, e
estávamos trabalhando cada vez melhor. Alugamos mais um salão ao lado do
nosso, que ficou vago, ampliamos a fábrica e em um curto espaço de
tempo alugamos mais dois salões com telefone: mas esses eram
embaixo, e os outros no primeiro andar. Minha filha Bethi saiu do
emprego e procurava por outro, então sugeri que trabalhasse
conosco. Enquanto isso, meu genro Rubem prestou concurso para dentista do governo do
estado, e passou a trabalhar na parte da manhã como dentista do serviço
público e à tarde no consultório dele. Algum tempo depois, vendeu seu
consultório e começou também a trabalhar conosco. Continuou
no emprego do estado durante a manhã, e depois vinha para a fábrica. Entrou 
como sócio, e minha filha voltou para casa para cuidar dos filhos e
administrar o lar.

Compramos máquinas modernas, e os negócios passaram a ficar excelentes. Fui com
meu filho para os Estados Unidos importar fios de \textit{nylon}, até
licença para importar conseguimos. Isso tudo se passou durante 1961. 
Meu filho voltou para o Brasil, mas eu viajei para ver os
meus irmãos em Austin. Fiquei mais vinte dias com eles, e depois fui a
Nova York fazer negócios com uma grande empresa, a Tai-cara.

\textls[10]{Eles ficaram de me responder. Enviariam uma carta para Austin, para poderem 
coletar informações. Eu já tinha feito negócios nos Estados
Unidos quando morava no Recife, e meu nome constava como importador.
Alguns dias mais tarde, chegou a carta, dizendo que não
existiam impedimentos e que poderíamos seguir. }

A carta estava
escrita em inglês, meu sobrinho leu e começou a rir. Aproximou-se, me
beijou e perguntou: ``Tio, o senhor sabe com quem estava falando na Tai-cara? Com
o principal presidente da companhia! Estou orgulhoso do senhor e do
meu pai também, vocês fazem grandes negócios. E você, tio, nem sabe
falar e escrever em inglês e consegue administrar tudo isso. Orgulho-me
de ter um tio tão capaz''.\looseness=-1

Fiquei com meus irmãos e irmãs mais alguns dias. Ao voltar para Nova York, 
fui mais uma vez à Tai-cara e conversei com o presidente. Disse que,  
voltando ao Brasil, lhe escreveria. Precisava entender como importar os fios de \textit{nylon}. Preparava-me para voltar ao Brasil, quando ouvi pelo rádio que o presidente Jânio
Quadros tinha renunciado.

\textls[-10]{Chegando, observei o que estava acontecendo: não
deixavam o vice-presidente João Goulart assumir a presidência. 
Finalmente, depois de alguns dias ele conseguiu assumir, mas a crise já estava instaurada. Os bancos tinham sido fechados, não se fazia mais negócio e o comércio parou.}\looseness=-1

%\chapter{Negócios em família}

\textls[10]{Assim se passaram alguns meses, até que aos poucos começou a melhorar a
situação. Mas eu já não podia mais pensar em trazer o \textit{nylon} dos Estados
Unidos, o dólar tinha subido muito e não compensava pois o frete 
ficava muito alto.}

\textls[5]{Tivemos que desistir da importação. Meu filho Elias, lendo uma
notícia no jornal, viu que estavam abrindo uma fábrica de fio
de \textit{agilon} em São Paulo. Ele foi conversar com os proprietários, que nos venderam o fio. Passamos a usar em nossas máquinas, produzimos uma
quantidade de tecido e mandamos tingir.}

\textls[10]{Encontramos um jeito de fabricar tecido. Fomos os primeiros no Brasil a usar do método. Fabricamos camisas para homens e blusas para
senhoras, casacos e roupas para crianças, roupas prontas que eram
distribuídas por vendedores nas cidades do interior. Tiveram boa
aceitação no mercado de confecção, devido à qualidade e ao preço. Além
disso, fizemos uma propaganda muito boa.}

\chapter{Tempos de crise}

\textls[-15]{Mesmo assim, perdemos muito dinheiro por causa da
crise. Muitos comerciantes deixaram de nos pagar o que estavam devendo, e
outros mandavam de volta as mercadorias. Durante alguns anos, para
completar, não fez muito frio, e as peças de lã que fizemos para o
inverno, acabaram sendo vendidas com preço muito abaixo. Para complicar mais, tínhamos comprado uma máquina nos Estados Unidos,
pouco antes da crise, com dinheiro emprestado do banco. Participamos da exposição da \textsc{fenit}, com mostruário fabricado, e gastamos também muito dinheiro com isso.}\looseness=-1

\textls[-10]{A crise apertou, e começou a faltar \textit{nylon} novamente. Fizemos um
contrato com a fábrica que produzia o fio, dizendo que precisávamos
de três toneladas do \textit{agilon}. Eles mandavam até mais, pois fomos os primeiros a descobrir como se produzia o tecido, e nós aceitávamos. Trabalhávamos muito e fazíamos propaganda, quando o
artigo se tornava conhecido vendíamos bastante. Tínhamos vendedores em diversas
cidades, que faziam pedidos das mercadorias e mandavam para a fábrica.}\looseness=-1

\textls[10]{Justamente na época de crise, o senhor Morelli, principal proprietário da empresa onde
eu comprava fio, viajou para a Europa. O senhor que
ocupou seu lugar junto aos acionistas da empresa (eles 
não compareciam presencialmente) passou a dar as ordens, e resolveu
começar a nos fornecer uma quantidade menor do \textit{agilon}. }

\textls[-10]{O prazo de pagamento era de 45--60--90 dias, mas quando começou a
faltar fios de \textit{banlon} e \textit{helanca}, a fábrica de \textit{agilon} passou a ficar
conhecida e os fabricantes que precisavam de fio corriam para lá
oferecendo pagamento à vista. A fábrica se aproveitou da ocasião e descontavam os cheques no mesmo dia. Ganharam muito dinheiro, e enviavam os fios de dez a 15 dias depois, às vezes até vinte. Começaram a nos fornecer menos fio, uma quantidade cada vez menor até
chegar a 600 quilos por mês, e depois 400. Acabei por ficar com a
fábrica parada por falta de material.}\looseness=-1

\textls[5]{Precisei dispensar muitos funcionários e operários que estavam 
comigo há muito tempo. As leis do Brasil obrigam a pagar todos os
direitos do trabalhador quando ele é dispensado do serviço, tudo tem que ser calculado com base no tempo de trabalho exercido.
Muitos trabalharam comigo durante vários anos, e eu tive que pagar muito
dinheiro para eles. A crise foi se prolongando, e a cada vez a situação se
tornava pior. Eu já havia comprado lã para o inverno, como
sempre com antecedência de quatro a seis meses, para fabricar e vender quando começasse o frio. Mas novamente naquele ano o frio não veio.
Fiquei com a lã preparada sem vender, e fui obrigado a comprar \textit{helanca} ou
\textit{banlon}. Mas não direto da fábrica, e sim de revendedores que cobravam mais
pelo fio, e isso encareceu minha produção.}

Tudo foi piorando progressivamente, e os comerciantes deixaram de pagar as
mercadorias já enviadas. Tinha ainda muito dinheiro para
receber, e resolvi viajar para cidades do interior onde moravam os comerciantes que me
deviam muito, e tentar cobrar. Demorei 15 dias, mas consegui receber
um pouco do dinheiro, e mandei para os meus filhos. Pedi que pagassem aos poucos o que nós devíamos, assim nossas dívidas
diminuiriam. Mas o dinheiro que mandei não foi suficiente para pagar em dia
as contas. Os fornecedores começaram a ameaçar entregar
nossas duplicatas assinadas ao cartório de protesto.

\textls[-13]{Enquanto isso acontecia, eu não estava em São Paulo. Meu filho e meu
genro ficaram assustados e foram consultar um advogado. Contaram que tínhamos muita mercadoria estocada,
mas não tínhamos dinheiro no momento para pagar, a não ser as
máquinas. Estávamos lidando com muita gente, e precisávamos ser aconselhados 
sobre o melhor a fazer. O advogado nos disse que
devíamos pedir concordata, para evitar que fossemos protestados. Pedimos, e foi aceito: pagaríamos só 60\% da dívida. Essa era a lei no Brasil, não queríamos tomar
nada de ninguém, apenas pagar todas as nossas dívidas. Meu filho e
meu genro declararam dessa forma na justiça, e dessa maneira não poderiam nos
protestar em cartório.}\looseness=-1

\textls[10]{O processo demorou 18 meses para sair da justiça, e teríamos
três anos para pagar nossas dívidas, em duas vezes: uma metade em um ano, e a
outra em dois anos. Voltando de viagem, encontrei a solução executada, sem poder fazer mais nada. Fui até os fornecedores falar sobre nossas
máquinas e mercadorias em estoque. Disse para eles: ``Vocês não 
terão que esperar tanto tempo, o processo já está na justiça. Preciso apenas dessa ordem para poder vender minhas máquinas e mercadorias. Assim que tiver, 
poderei pagar''.}

\chapter{De cabeça erguida}

No dia 24 de fevereiro de 1964, saiu no jornal o valor que eu estava
devendo para cada um dos fabricantes. Depois, começou o processo que 
levaria 18 meses até que o juiz dividisse a dívida em duas vezes.

Resolvi que não faria os credores esperarem tanto para receber
o dinheiro. Procurei por eles e disse que queria pagar antes. Ofereci que
aceitassem minhas mercadorias e máquinas pela dívida. Se concordassem, pediria a expedição de uma ordem para pôr em prática meu plano. E então resolveria tudo na revogação da
concordata. Chamei o advogado, que logo entrou em contato com o juiz, e explicou minha vontade. O juiz aceitou e anulou a concordata, e eu aos
poucos paguei as dívidas assumidas com as máquinas e as mercadorias vendidas.
Com o que recebi pelas máquinas paguei também nossos operários, e aos
poucos os despedi. Eles receberam tudo que lhes era devido. 

\textls[10]{Aluguei um
pequeno salão para diminuir as despesas e convidei para trabalhar
conosco somente o sogro do
meu filho, que já estava colaborando há algum tempo. Também deixei só um funcionário trabalhando. Aproveitamos as
máquinas que ainda não tínhamos entregado e produzimos tecidos para
terceiros. Recebíamos pelo nosso trabalho, e isso ajudou a pagar mais
algumas dívidas.}

\textls[10]{Meu filho e meu genro começaram a procurar outros serviços, porque não
dava para cobrir as despesas. Minha filha começou a
trabalhar como coordenadora na Ofidas, a Organização Feminina Israelita de Assistência Social. Tinham lá um Lar das Crianças, onde cuidavam de 70 crianças, filhos de mães
pobres que precisavam trabalhar e deixavam os filhos durante o dia.}

\textls[10]{Meu genro voltou a trabalhar como dentista. Meu filho foi
trabalhar em uma loja de carros usados, e ganhava comissão sobre as
vendas. A esposa dele também passou a trabalhar na Ofidas,
supervisionando os livros de contabilidade.}

Mas voltando um pouco para minha história, quero contar o que houve quando fui
conversar com os proprietários das fábricas. Eles me perguntaram por que
eu não tinha contado sobre o que estava acontecendo com o
nosso negócio. E perceberam que tinham destruído o que eu tinha
conseguido construir em 42 anos, mas já era tarde.

\textls[-15]{Perceberam que eu não era uma pessoa que ansiava por ficar com o que era
dos outros. Depois que entramos na justiça, se transformaram em \textit{boas pessoas}, e disseram que teriam me ajudado a pagar os
que nos ameaçavam de protestar. Disseram: ``Nós sabíamos que você era um
homem honesto e decente, que não enganaria a nós nem aos outros.
Conhecemos o seu passado, retidão e palavra''. Foi por isso que os fabricantes teriam dado um crédito tão grande para nós, mas essa
conversa foi feita muito tarde. Dali a alguns dias terminaria de pagar
tudo. Paguei a cada um deles no prazo de um ano o que teria demorado
mais de três anos e meio.}\looseness=-1

\textls[-7]{Fiquei sem nada, mas, graças a Deus, com minha honra intacta, que é
o maior tesouro da minha vida. Foi meu pai, que Deus o tenha em
paz, que me disse isso ao nos despedirmos em Varsóvia na última
vez que nos vimos, em junho de 1922. Nunca mais o encontrei na vida. Ele me disse: 
``Meu filho, você está partindo para longe, o caminho é distante, e até
conseguir se reunir com seus irmãos estará só. Procure ficar sempre
entre os judeus. Onde você estiver, honre o seu nome, e você nunca
perecerá''. Lembrei e segui até hoje suas palavras. E durante todos os 42 anos desde nossa despedida, mantive a
postura honrada e respeitosa, e resguardei o meu nome em todos os países
onde vivi. Posso andar de cabeça erguida por todos os lugares pelos quais 
passei, e ninguém pode dizer, apontando o dedo, que me apropriei de
algo que pertencia a mais alguém.}\looseness=-1
