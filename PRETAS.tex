\textbf{Chaim Novodvorsky} (Knishin, Polônia, 1902--São Paulo, 1983) \textls[-10]{deixou sua família em um pequeno vilarejo polonês aos 19 anos com o objetivo de imigrar para os Estados Unidos: parte dos irmãos já estava na América e ele fugia do serviço militar, um dos mais frequentes motivos de emigração da Europa Oriental e do Império Russo desde o século \textsc{xix}. Diante da recusa de visto, lançou-se de toda forma rumo ao outro lado do mundo e trabalhou nas mais diversas profissões --- como padeiro, mascate, leitor de realejo, comerciante, lojista, importador, fabricante. Além disso, construiu uma história de vida que perpassa temas importantes da história judaica e sul-americana, como a passagem por Moisés Ville, colônia agrícola fundada pela Jewish Colonization Association na Argentina; a atuação no teatro iídiche; a participação ativa nas instituições comunitárias; e a doação de aviões à recém-criada Força Aérea Brasileira durante a \textsc{ii} Guerra Mundial.}

\textbf{Em busca de meus irmãos na América} é um relato pessoal singular e ao mesmo tempo emblemático dos percursos da imigração judaica entre os anos 1920 e 1960. Novodvorsky, que imigrou primeiro para a Argentina, depois para o Uruguai e, finalmente, para o Brasil, construiu esse roteiro longo em função de um objetivo final e aspirado, que era o deslocamento para os Estados Unidos. Suas memórias também revelam fatores estruturais que caracterizaram o estabelecimento dos judeus no Brasil, como as barreiras impostas à imigração em países como Estados Unidos e Canadá durante a década de 1920, que fizeram do país um horizonte possível e desejável.

\pagebreak

\textbf{Lilian Starobinas} \textls[-10]{é neta de Chaim Novodvorsky. Historiadora e doutora em educação pela \textsc{usp}, é professora de história na Escola Vera Cruz, onde leciona também no curso de pedagogia. Junto a vários autores, publicou \textit{Vanguarda pedagógica: o legado do Ginásio Israelita Brasileiro Scholem Aleichem}, além de diversos artigos sobre a presença judaica no Brasil e teatro ídiche. É membro da coordenação dos coletivos Círculo de Reflexão sobre Judaísmo Contemporâneo e Trupe Ídiche, ambos na Casa do Povo.}\looseness=-1

\textbf{Léa Baran} (Rechitsa, Bielorrussia, 1927--São Paulo, 2016) foi professora, além de tradutora do iídiche e do russo.

\textbf{Roney Cytrynowicz} é historiador, doutor em história pela \textsc{usp}, diretor da Narrativa Um e autor, entre outros, de \textit{Guerra sem guerra: a mobilização e o cotidiano em São Paulo durante a Segunda Guerra Mundial} (2000) e \textit{Memória da barbárie: a história do genocídio dos judeus} (1990), ambos publicados pela Edusp. Foi diretor de acervo do Arquivo Histórico Judaico Brasileiro.\looseness=-1




